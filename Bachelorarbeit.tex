\documentclass[DIN, pagenumber=false, fontsize=11pt, parskip=half, colorinlistoftodos, svgnames]{scrartcl}
%\\
\usepackage[utf8]{inputenc}
%\usepackage[ngerman]{babel}
\usepackage[T1]{fontenc}

\usepackage{xeCJK} %why am I doing this...?
\xeCJKsetup{CJKmath=true}
\setCJKmainfont{Noto Serif JP} 
\setCJKsansfont{Noto Sans CJK JP Light}
\setCJKmonofont{Noto Sans Mono CJK JP}


%\usepackage{cite}
\usepackage{enumitem}
%\usepackage{hyperref} %Das lädt automatisch...?

\usepackage[backend=biber, style=alphabetic ]{biblatex}
\addbibresource{bib/quellen.bib}

\usepackage{nchairx} %Lehrstuhl package
\usepackage{mathtools}			
\usepackage{amssymb}
%\usepackage{amsthm} %conlfict with nChairX
\usepackage{mathabx} %for dashV etc. conflict with amssymb
\usepackage{eurosym}
\usepackage{leftindex} 
\usepackage{graphicx}
\usepackage{pgf}
\usepackage{tikz}
\usetikzlibrary{cd}
\usepackage{typearea}
\usepackage[]{scrlayer-scrpage} %scrpage2 is obsolete and not supported by MikTex and TeX live 
\usepackage{subcaption}
\usepackage{lastpage}
\usepackage{multicol} %??
\usepackage{float}
\usepackage[hidelinks]{hyperref}
%\usepackage[table]{xcolor} %Übernommen aus Protokollen letzten Jahres, Nutzen unbekannt:
%\usepackage[margin=10pt,font=small,labelfont=bf]{caption}
\usepackage{textgreek}
\usepackage{marvosym} %Eurozeichen. mehr?

%LinLog (par and with, etc.)
\usepackage{cmll}

%Logic
\usepackage{bussproofs}

\usepackage{todonotes} %to do notes...
\setuptodonotes{tickmarkheight=0.1cm}
%different types of todonotes:
\newcommand{\formatnote}[2][]{\todo[color=cyan!40, #1]{#2}}
\newcommand{\questionnote}[2][]{\todo[color=purple!80, #1]{#2}}
\newcommand{\urgentnote}[2][]{\todo[color=red, #1]{#2}}
\newcommand{\notimportantnote}[2][]{\todo[color=LightPink, #1]{#2}}



\usepackage[a4paper,margin=3cm]{geometry}

%definition, theorem, remark etc. (already in nchairX)
%\theoremstyle{definition}
%\newtheorem{definition}{Definition}[section]



%\addto\captionsngerman{\renewcommand\figurename{Abb.}}
%\addto\captionsngerman{\renewcommand\tablename{Tab.}}

%\usepackage{siunitx} %SI-Einheiten. clash with nchairx (\unit)
%\DeclareSIUnit[]{\euro}{\text{\EUR}}

%Nützliche Shortcuts:
\def\SP#1{\textsuperscript{{#1}}} % Index oben, textmode
\def\SB#1{\textsubscript{{#1}}} % Index unten, textmode
\def\SPSB#1#2{\rlap{\SP{#1}}\SB{#2}} % Kombi Index oben und unten, textmode
\def\QR#1#2{\raisebox{1ex}{\ensuremath{#1}}\ensuremath{\mkern-3mu}\big/\ensuremath{\mkern-3mu}\raisebox{-1ex}{\ensuremath{#2}}} % schräger Bruch
\newcommand{\dd}[1]{\mathrm{d}#1} % kleines d Diff.op
\def\Ree{\operatorname{Re}} % real part Re (not fraktur R)
\def\Imm{\operatorname{Im}} % imaginary part Im (not fraktur I)
%\DeclareMathOperator{\tr}{tr} % Träger (already in nChairX)

%some stuff für Beweise mit mehreren Teilen:
\newcommand{\implication}[2]{%
	\enquote{\mbox{$\text{#1}\implies\text{#2}$}}%
	\enspace\ignorespaces
}
\newcommand{\statement}[1]{%
	#1\enspace\ignorespaces
}


\usepackage{csvsimple} %Import von csv-Dateien



%==============================================================================


\title{LinLog und LinDisCats}
\author{Max Wiegand}

%\sisetup{output-decimal-marker = {,}, per-mode = fraction, separate-uncertainty, exponent-product = \cdot,}
%\renewcommand{\thesection}{}
%\renewcommand\thesubsection{\Alph{subsection}.)}
\allowdisplaybreaks

%=============================================================================
%text starts 

\begin{document}
	\maketitle
	
	\section{Introduction}
	
	\todo[inline]{what am I even writing about?}
	
	\section{LinLog Preliminaries}
	
	\subsection{Motivation/Intuition}
	\label{sec: intuition}
	
	%................................
	\todo[inline, color=green!50]{feels kinda rambling... combine intuition with syntax definition below}
	%comment..............................................................................
	
	Classical (and intuitionistic) logic deals with the propagation of stable truth values. If one has a true sentence $A$ and an an implication $A \Rightarrow B$, then $B$ follows while $A$ remains true. However, real-life implications are often causal and modify their premises. They cannot therefore be iterated arbitrarily. For example if $A$ describes the ownership of 1\euro\ and $B$ owning a chocolate bar, an implication $A \multimap B$ (to be formally introduced later) would describe the process of buying such a chocolate bar for 1\euro, losing the 1\euro\ in the process.
	
	While such dealings with resources can of course be modeled in classical logic, it is easier done in the resource-sensitive \emph{linear logic}, first described by Girard in 1987 \cite{girard87}. 
	
	Here, we have two conjunctions simultaneously $\otimes$ ("times" or "tensor") and $\with$ ("with"), which describe the availability of resources:
	
	Suppose $C$ is the ownership of a cookie and it costs also 1\euro\ (i.e. we have $A\multimap C$). Then $B \otimes C$ states that one owns both a chocolate bar and a cookie. The implication $A \multimap B \otimes C$ is not possible, as it would mean that you are buying both, cookie and chocolate bar at the same time, for just 1\euro\ total. However, from $A \multimap B$ and $A\multimap C$ we get $A \otimes A \multimap B \otimes C$, i.e. the process of buying both for 2\euro.
	
	On the other hand, $B \with C$ states that one has a choice between either $B$ or $C$ (imagine a token). From the implications $A\multimap B$ and $A\multimap C$ we get the implication $A \multimap B \with C$, i.e. the process of buying a token to be exchanged for a chocolate bar or a cookie at a later time (with the choice lying with oneself). While this may  seem like a disjunction, both implications $B\with C \multimap B$ and $B \with C \multimap C$ (exchanging the token for either product) are provable from $B \with C$, although not simultaneously. 
	
	Dually, we have two disjunctions $\parr$ ("par") and $\oplus$ ("plus"):
	
	Suppose now that $B$ and $C$ are the ownership of a figurine of Pikachu or Mew respectively. Then $B \oplus C$ may be the ownership of a Kinder Egg containing either figurine. This means when buying that egg ($A \multimap B \oplus C$) we do not know which one we will get. 
	
	Our second disjunction, dual to $\otimes$, can be understood by linear implication and the linear negation (denoted as $(\cdot)^\bot$): Under the interpretation of ownership the linear negation is no interpreted as the absence of ownership but as negative ownership, i.e. debt. That means the negation of owning 1\euro, $A$, is owing someone 1\euro, $A^\bot$. With the par operator we can now write the linear implication $A \multimap B$ symmetrically as $A^\bot \parr B$. 
	\todo[inline]{(explicit interpretation?! shared pool between people containing Pika and Mew -> you have to get rid of one to use the other?)}
	%Interpretation von par...............................................................................................................................
	
	In order to regain our stable truths known from classical logic, we need to employ two unitary connectives $\oc$ ("of course" or "bang") and $\wn$ ("why not"). The bang operator informs us that there is an infinite amount of a resource: The statement $\oc A$ translates into the ownership of an amount of money that is large enough for us to ignore resource sensitivity. Imagine for example a billionaire buying a Pokemon figurine instead of a social media site: His amount of money will not be noticeably smaller after buying the figurine. We can informally say $\oc A = (1 \with A) \otimes (1 \with A) \otimes \cdots $ and therefore view classical logic as some sort of limit of linear logic, just as classical mechanics is a limit of quantum mechanics and the theory of relativity. 
	
	%......................................................
	\todo[inline]{interpretation of ?wn.}
	\todo[inline ]{interpretation of constants}
	%comment........................................................................................................................................
	
	
	%=============================================================================================================================
	\subsection{Syntax}
	\todo[inline]{section needs serious reformatting... }
	\todo[inline]{main source: IntroLinLog}
	
	Formulas are built from atomic 
	\urgentnote{formula? term? look up basic terminology}
	%used "formula" an unterschiedlichen Stellen
	formulas $p, q, \phi, \psi, p^\bot$ etc. and constants $1, \bot, 0, \top$ with connectives $\otimes, \parr, \with, \oplus, \oc, \wn$.
	
	Let $\Gamma $, $\Delta$ etc. be arbitrary, finite lists of formulas (e.g.: $\Gamma = (p_1, ..., p_n)$) and $A$ and $B$ formulas. %clunky sentence
	As we will later consider fragments without negation, we shall define linear logic with a two-sided calculus. For structural rules we only have the exchange rule, missing the (general) weakening and contraction rules known from classical logic:
	
	\begin{center}
		\AxiomC{$\Gamma, A, B, \Gamma' \vdash \Delta$}
		\RightLabel{ex.L}
		\UnaryInfC{$\Gamma, B, A, \Gamma' \vdash \Delta$}
		\DisplayProof
		\quad
		\AxiomC{$\Gamma \vdash \Delta , A, B, \Delta'$}
		\RightLabel{ex.R}
		\UnaryInfC{$\Gamma \vdash \Delta , B, A, \Delta' $}
		\DisplayProof
	\end{center}
	
	
	Obviously, we have an identity and a cut rule:
	%Erklaerung der importance von Cut?
	%anderes Wort als obviously!....................................................................................................................
	
	\begin{center}
		\AxiomC{\vphantom{$\frac{a}{b}$} } %temporäre Lösung (auch an anderen Stellen verwendet?)
		\RightLabel{id}
		\UnaryInfC{$A \vdash A$}
		\DisplayProof
		\quad
		\AxiomC{$\Gamma \vdash \Delta , A$}
		\AxiomC{$A , \Gamma' \vdash \Delta' $}
		\RightLabel{Cut}
		\BinaryInfC{$\Gamma , \Gamma' \vdash \Delta , \Delta' $}
		\DisplayProof
	\end{center}
	\notimportantnote{permanente Lösung für empty axiom finden}
	
	
	As already mentioned, the classical conjunction $\wedge$ and disjunction $\vee$ split into two respectively. These can be classified as multiplicative and additive connectives. The calculus rules for the \emph{multiplicative} conjunction $\otimes$ and disjunction $\parr$ ("par") are as follows:
	
	\begin{center}
		\AxiomC{$\Gamma , A, B \vdash \Delta$}
		\RightLabel{$\otimes_L$}
		\UnaryInfC{$\Gamma A\otimes B \vdash \Delta $}
		\DisplayProof
		\quad
		\AxiomC{$\Gamma \vdash \Delta , A$}
		\AxiomC{$\Gamma' \vdash B, \Delta'$}
		\RightLabel{$\otimes_R$}
		\BinaryInfC{$\Gamma, \Gamma' \vdash \Delta , A\otimes B, \Delta'$}
		\DisplayProof
		
		
		\AxiomC{$\Gamma , A \vdash \Delta $}
		\AxiomC{$\Gamma' , B \vdash \Delta' $}
		\RightLabel{$\parr_L$}
		\BinaryInfC{$\Gamma, A\parr B, \Gamma' \vdash \Delta, \Delta'$}
		\DisplayProof
		\quad
		\AxiomC{$\Gamma \vdash \Delta , A, B$}
		\RightLabel{$\parr_R$}
		\UnaryInfC{$\Gamma \vdash \Delta , A \parr B $}
		\DisplayProof
	\end{center}
	%Bedeutung
	
	
	\begin{remark}
		The rules $\otimes_L$ and $\parr_R$ imply, that the commas are to be read as $\otimes$ on the left-hand side and as $\parr$ on the the right-hand side. That means $A, B \vdash C, D$ is provable iff $A \otimes B \vdash C \parr D$ is provable. 
	\end{remark}
	
	\begin{proof}
		We only have to show that $A , B \vdash C , D $ follows from $A\otimes B \vdash C \parr D$ as the other direction is just our introduction rule:
		
		\begin{center}
			\AxiomC{$A \otimes B \vdash C \parr D $}
			
			\AxiomC{}
			\RightLabel{id}
			\UnaryInfC{$A \vdash A $}
			\AxiomC{}
			\RightLabel{id}
			\UnaryInfC{$B \vdash B $}
			\RightLabel{$\otimes_R$}
			\BinaryInfC{$A , B \vdash A \otimes B $}
			\RightLabel{Cut}
			\BinaryInfC{$A , B \vdash C \parr D $}
			
			\AxiomC{}
			\UnaryInfC{$C \vdash C $}
			\AxiomC{}
			\UnaryInfC{$D \vdash D $}
			\RightLabel{$\parr_L $}
			\BinaryInfC{$C \parr D \vdash C , D $}
			\BinaryInfC{$A  , B \vdash C , D$}
			\DisplayProof
		\end{center}
	\end{proof}
	\formatnote[]{proofsymbol :/ }
	
	The calculus rules for the \emph{additive} conjunction $\with$ and disjunction $\oplus$ ("plus") are as follows:
	
	\begin{center}
		\AxiomC{$\Gamma , A \vdash \Delta $}
		\RightLabel{$\with_{L1}$}
		\UnaryInfC{$\Gamma , A \with B \vdash \Delta $}
		\DisplayProof
		\quad
		\AxiomC{$\Gamma , B \vdash \Delta $}
		\RightLabel{$\with_{L2}$}
		\UnaryInfC{$\Gamma , A \with B \vdash \Delta $}
		\DisplayProof
		
		\AxiomC{$\Gamma \vdash \Delta , A $}
		\AxiomC{$\Gamma \vdash \Delta , B $}
		\RightLabel{$\with_R$}
		\BinaryInfC{$\Gamma \vdash \Delta , A \with B $}
		\DisplayProof
	\end{center}
	
	\begin{center}
		\AxiomC{$\Gamma , A \vdash \Delta $}
		\AxiomC{$ \Gamma , B \vdash \Delta $}
		\RightLabel{$\oplus_L$}
		\BinaryInfC{$\Gamma , A\oplus B \vdash \Delta$}
		\DisplayProof
		
		\AxiomC{$\Gamma \vdash A , \Delta $}
		\RightLabel{$\oplus_{R1}$}
		\UnaryInfC{$\Gamma \vdash A \oplus B , \Delta $}
		\DisplayProof
		\quad
		\AxiomC{$\Gamma \vdash B , \Delta $}
		\RightLabel{$\oplus_{R2}$}
		\UnaryInfC{$\Gamma \vdash A \oplus B ,  \Delta $}
		\DisplayProof
	\end{center}
	
	Notice the difference between $\with_R$ and $\otimes_R$ (and dually between $\oplus_L$ and $\parr_L$): while for $\otimes_R$ the contexts $\Gamma$ etc. are arbitrary and get combined in the conclusion, $\with_R$ requires the contexts to be equal. In classical logic these rules would be equivalent using its additional structural rules.
	
	\begin{remark}
		Similarly to the multiplicative connectives above, the additive connectives have a invertibility statement: $\Gamma \vdash A \with B $ is provable iff $\Gamma \vdash A$ and $\Gamma \vdash B$ are provable. (The statement for $\oplus$ is formulated dually)
	\end{remark}
	
	\begin{proof}
		As with the multiplicative statement, one direction suffices:
		\begin{center}
			\AxiomC{$\Gamma \vdash A \with B$}
			
			\AxiomC{}
			\UnaryInfC{$A \vdash A$}
			\UnaryInfC{$A \with B \vdash A$}
			\BinaryInfC{$\Gamma \vdash A$}
			\DisplayProof
		\end{center}
		$\Gamma \vdash B$ follows the same way.
	\end{proof}
	
	\todo[inline]{above remark unnötig?}
	
	We now define the linear negation $(\cdot)^\bot$ as follows:
	
	\begin{itemize}
		\item For the constants: 
		\begin{align*}
			1^\bot := \bot \quad 
			& 
			\bot^\bot := 1 
			\\
			\top^\bot := 0 \quad 
			& 
			0^\bot := \top
		\end{align*}
		\item For atomic formulas: the negation of $p$ is $p^\bot$. The negation $\left(p^\bot\right)^\bot$ of $p^\bot $ is $p$.
		\item For non-atomic formulas we define the negation by the De Morgan equations:
		\begin{align*}
			\left(A \otimes B\right)^\bot := A^\bot \parr B^\bot 
			\quad
			&
			\left(A \parr B\right)^\bot := A^\bot \otimes B^\bot 
			\\
			\left(A \with B\right)^\bot := A^\bot \oplus B^\bot 
			\quad
			&
			\left(A \oplus B\right)^\bot := A^\bot \with B^\bot 
		\end{align*}
		\item We define linear implication with the par-operator: 
		\begin{align*}
			A \multimap B := A^\bot \parr B
		\end{align*}
		\item As with classical logic, we translate a two-sided sequent 
		$A_1, ..., A_n \vdash B_1, ..., B_n$ 
		into a right-sided sequent 
		$\vdash A_1^\bot, ... A_n^\bot, B_1, ... B_n$ 
		by negation of the left-hand side and vice versa.
		With this it is easily seen that $\vdash A\multimap B$ iff $A \vdash B$.
		
	\end{itemize}
	
	With this our calculus rules above become quite redundant and we can restrict them on the rules for the right-hand side. However, these redundancies become necessary when dealing with a negation-free fragment of LL.
	
	The modality connectives reintroduce stable truths and with them the the weakening and contraction rules known from classical logic:
	
	\begin{center}
		\AxiomC{$\vdash \wn\Gamma , A$}
		\RightLabel{$\oc$}
		\UnaryInfC{$\vdash \wn\Gamma , \oc A$}
		\DisplayProof
		\quad
		\AxiomC{$\vdash \Gamma$}
		\RightLabel{$\wn_W$ (weakening)}
		\UnaryInfC{$\vdash \Gamma , \wn A$}
		\DisplayProof
		
		\AxiomC{$\vdash \Gamma , A$}
		\RightLabel{$\wn_D$ (dereliction)}
		\UnaryInfC{$\vdash \Gamma , \wn A$}\
		\DisplayProof
		\quad
		\AxiomC{$\vdash \Gamma , \wn A , \wn A $}
		\RightLabel{$\wn_C$ (contraction)}
		\UnaryInfC{$\vdash \Gamma , \wn A$}
		\DisplayProof
		
		\begin{align*}
			\left(\oc A \right) ^\bot = \wn \left( A^\bot \right) 
			\quad 
			\left(\wn A \right) ^\bot = \oc \left( A^\bot \right) 
		\end{align*}
	\end{center}
	
	\notimportantnote[noline]{Definition of $\wn \Gamma $ korrekt?}
	Here the Kontext $\wn \Gamma $ is given by applying the $\wn $-modality to every formula in the list of $\Gamma $, i.e. $\wn \Gamma = \wn q, \wn p, \ldots $ for $\Gamma = q, p \ldots $
	
	\begin{remark}
		These modalities are called exponentials because of the
		\urgentnote{definition linear equivalence}
		following relation:
		\begin{align*}
			\oc\left(A \with B\right) \equiv \oc A \otimes \oc B ,
			\quad 
			\wn \left(A \oplus B \right) \equiv \wn A \parr \wn B
		\end{align*}
	\end{remark}
	
	\begin{proof}
		We will only prove the second equivalence. The first is acquired by duality.
		
		\begin{center}
			\AxiomC{$\vdash A, A^\bot $}
			\RightLabel{$\oplus $}
			\UnaryInfC{$A^\bot, A\oplus B$}
			\RightLabel{$\wn_D $}
			\UnaryInfC{$\vdash A^\bot, \wn (A\oplus B) $}
			\RightLabel{$\oc $}
			\UnaryInfC{$\vdash\oc (A^\bot), \wn (A\oplus B) $}
			
			\AxiomC{$\vdash B, B^\bot$}
			\RightLabel{$\oplus $}
			\UnaryInfC{$\vdash B^\bot, A\oplus B$}
			\RightLabel{$\wn_D $}
			\UnaryInfC{$\vdash B^\bot, \wn (A\oplus B) $}
			\RightLabel{$\oc $}
			\UnaryInfC{$\vdash \oc (B^\bot), \wn (A\oplus B) $}
			
			\RightLabel{$\otimes $}
			\BinaryInfC{$\vdash \oc(A^\bot) \otimes \oc (B^\bot) , \wn (A\oplus B) , \wn (A\oplus B) $}
			\RightLabel{$\wn_D $}
			\UnaryInfC{$\vdash \oc(A^\bot) \otimes \oc (B^\bot) , \wn (A\oplus B) $}
			\RightLabel{$(-)^\bot$}
			\UnaryInfC{$\vdash (\wn A \parr \wn B)^\bot , \wn (A \oplus B) $}
			\UnaryInfC{$\vdash \wn A \parr \wn B \multimap \wn (A \oplus B)$}
			\DisplayProof
			
			
			
			\AxiomC{$\vdash A^\bot , A$}
			\UnaryInfC{$\vdash A^\bot, \wn A$}
			\RightLabel{$\wn_W$}
			\UnaryInfC{$\vdash A^\bot, \wn A, \wn B $}
			
			\AxiomC{$\vdash B^\bot, B$}
			\UnaryInfC{$\vdash B^\bot, B$}
			\RightLabel{$\wn_W$}
			\UnaryInfC{$\vdash B^\bot, \wn B , \wn A $}
			
			\RightLabel{$\with $}
			\BinaryInfC{$\vdash A^\bot \with B^\bot , \wn A , \wn B $}
			\UnaryInfC{$\vdash \oc (A^\bot \with B^\bot) , \wn A, \wn B $}
			\UnaryInfC{$\vdash \oc (A^\bot \with B^\bot) , \wn A \parr \wn B $}
			\UnaryInfC{$\vdash \wn (A \oplus B) \multimap \wn A \parr \wn B $}
			\DisplayProof
		\end{center}
	\end{proof}
	
	\formatnote{proofsymbol なんで? :( }
	
	%.......................................
	\notimportantnote[inline]{MAYBE to be added: \textbackslash forall and \textbackslash exist. Do we want predicate logic?}
	%comment..............................................................................................
	
	
	
	%=============================================================================================================
	\subsection{Blabla proof-structures and proof-nets}
	
	We shall now leave behind any capitalistic resource interpretation and concentrate on what linear logic was invented for: analyzing proofs.
	
	\todo[inline]{to be continued... }
	\todo[inline]{how to typeset axiom links? :(}
	
	
	
	%___________________________________________________________________________________________________________________________________________________________________________________________________________________________________________________________________________________________
	\section{Categorical Preliminaries}
	
	\subsection{Categories}
	\label{sec: catSec}
	%Tensorcats?
	
	\todo[inline]{sources for basic definitions. (nlab prolly doesn't suffice)}
	
	\begin{definition}[Category]
		\label{def: cat}
		A Category $\category{C}$ consists of the following data: 
		\begin{itemize}
			\item A class $\Obj(\category{C})$ of \emph{objects}.
			\item For every pair of objects $A, B \in \Obj(\category{C})$ there is a class $\Hom(A, B)$ of \emph{morphisms} $f: A \rightarrow B$ from $A$ to $B$. 
			\item Morphisms \emph{compose}: For $f \in \Hom(A, B)$ and $g \in \Hom(B, C)$ there is a morphism $g \circ  f \in \Hom(A, C)$. 
			That composition is associative: 
			\begin{align*}
				h \circ (g \circ f) = (h \circ g) \circ f
			\end{align*}
			We will write $gf$ for $g\circ f$ when appropriate.
			\item For every object $A$ there is an \emph{identity morphism} $\id_A \in \Hom(A, A)$:
			\begin{align*}
				f \circ \id_A = f, \quad \id_B \circ f = f
			\end{align*}
			for $f \in \Hom(A, B)$
		\end{itemize}
		If the hom-classes are sets, we call the category \emph{locally small}. If the object class is a set, we call the category \emph{small}. Otherwise, we call the category \emph{large}.
		
		If we have a category $\category{C} $, we call its \emph{opposite} category $\category{C}^\opp $ the category with the following structure:
		\begin{itemize}
			\item $\Obj{\left(\category{C}^\opp \right)} = \Obj(\category{C}) $
			\item $\forall A, B \in \Obj(\category{C}): 
			\Hom_{\category{C}^\opp }(A,B) = \Hom_{\category{C}}(B,A)$
		\end{itemize}
	\end{definition}
	
	%.....................................................................................
	\notimportantnote[inline]{extra definition for opposite structure? }
	%commment....................................................................................
	
	
	\begin{definition}[Functor]
		\label{def: functor}
		A \emph{(covariant) functor} $\functor{F}: \category{C} \rightarrow \category{D} $ between two categories $\category{C}$ and $\category{D} $ is a function mapping each object $A \in  \Obj(\category{C}) $ to an object $F(A) \in \Obj(D) $ and each morphism $f \in \Hom(A,B) $ to a morphism $f \in \Hom(\functor{F}(A), \functor{F}(B))$ such that identity and composition are preserved:
		\begin{align*}
			\functor{F}(\id_A) = \id_{\functor{F}(A)},
			\quad
			\functor{F}(g\circ f) = \functor{F}(g) \circ \functor{F}(f)
		\end{align*}
		
		A functor $\functor{F}: \category{C}^\opp \rightarrow \category{D} $ is called \emph{contravariant} on $\category{C} $.
		
		We will drop the parentheses when appropriate.
	\end{definition}
	
	
	\begin{definition}[Natural transformation]
		\label{def: natTrans}
		A \emph{natural transformation} $\tau: F \rightarrow G $ between two functors $\functor{F}, \functor{G}: \category{C} \rightarrow \category{D} $ is family of morphisms in $\category{D}$:
		\begin{align*}
			\tau = \{ \tau_A : \functor{F}A \rightarrow \functor{G}A \,| A \in \Obj(\category{C})\}
		\end{align*}
		such that $\tau_B\functor{F}(f) = \functor{F}(f)\tau_A $ for all $f: A \rightarrow B \in \Morph(\category{C}) $, i.e. the following diagram commutes:
		
		\begin{center}
			\begin{tikzcd}
				\functor{F} A 
				\arrow[r, "\functor{F}f " ]
				\arrow[d, "\tau_A" ] 
				&
				\functor{F} B
				\arrow[d, "\tau_B" ]
				\\
				\functor{G} A 
				\arrow[r, "\functor{G}f" ]
				&
				\functor{G} B
			\end{tikzcd}
		\end{center}
		
		If $\tau_A$ is an isomorphism for all $A \in \Obj(\category{C}) $, we call $\tau$ a \emph{natural} \emph{isomorphism}.
		We will often represent a natural transformation by a single one of its members and also drop the index when appropriate.
	\end{definition}
	
	
	\todo[inline]{definition: equivalence of categories}
	

	
	\begin{definition}[Adjunction]
		\label{def: adjunction}
		\notimportantnote[]{source: Brandenburg}
		Let $\category{C}$ and $\category{D} $ be categories. We call $\functor{F}: \category{C} \rightarrow \category{D} $ the \emph{left adjoint} of $\functor{G}: \category{D} \rightarrow \category{C} $ and $\functor{G} $ the \emph{right adjoint} of $\functor{F} $ if there is an isomorphism that is natural in $A \in \category{C} $ and $B \in \category{D} $:
		\begin{align*}
			\Hom_\category{D}(\functor{F}A, B) 
			\cong \Hom_\category{C}(A, \functor{G}B)
		\end{align*}
		Equivalently, we call $\functor{F} $ the left adjoint of $\functor{G} $ if there are natural transformations 
		\begin{align*}
			\eta &: \id_\category{C} \rightarrow \functor{G} \circ \functor{F}
			\\
			\varepsilon &: \functor{F} \circ \functor{G} \rightarrow \id_\category{D}
		\end{align*}
		fulfilling the following condition: 
		
		\begin{align*}
			\id_{\functor{F}A} 
			&= \varepsilon(\functor{F}A) \circ \functor{F}\eta(A)
			\\
			\id_{\functor{G}B} 
			&= \functor{G}\varepsilon(B) \circ \eta(\functor{G}B)
		\end{align*}
		\todo{maybe index Schreibweise?}
		
		i.e. the following diagrams commute:
		\begin{center}
			\begin{tikzcd}
				&
				\functor{F} \circ \functor{G} \circ \functor{F}
				\arrow[rd, "\varepsilon \bullet \functor{F} " ]
				&
				\\
				\functor{F}\id_\category{C}
				\arrow[ru, "\functor{F} \bullet \eta " ]
				\arrow[rr, equal, "\id" ]
				&
				&
				\id_\category{C} \circ \functor{F}
			\end{tikzcd}
			\begin{tikzcd}
				&
				\functor{G} \circ \functor{F} \circ \functor{G}
				\arrow[rd, "\functor{G} \bullet \varepsilon " ]
				&
				\\
				\id_\category{C} \circ \functor{G}
				\arrow[ru, "\eta \bullet \functor{G} "]\
				\arrow[rr, equal, "\id " ]
				&
				&
				\functor{G} \circ \id_\category{D}
			\end{tikzcd}
		\end{center}
		\todo[inline]{cut triangle diagrams or define $\bullet $ } %diagramme.............................................................................................................................................
		
		We write $F \dashv G$.
	\end{definition}
	
	
	\begin{definition}[Monoidal category]
		\label{def: monoCat}
		A \emph{monoidal category} $(\category{C}, \otimes, \Unit, \alpha, \lambda, \rho)$ is a category $\category{C}$ with the following additional data:
		\begin{itemize}
			\item a functor $\otimes : \category{C} \times \category{C} \rightarrow \category{C} $, called \emph{monoidal} or \emph{tensor product};
			\item an object $\Unit \in  \Obj(\category{C})$, called \emph{unit object};
			\item a natural isomorphism $\alpha: A \otimes (B \otimes C ) \rightarrow (A \otimes B) \otimes C$, called \emph{associator};
			\item a natural isomorphism $\lambda: \Unit \otimes A \rightarrow A$, called left unitor;
			\item a natural isomorphism $\rho: A \otimes \Unit \rightarrow A$, called right unitor;
		\end{itemize}
		such that the following diagrams commute:
		\begin{itemize}
			\item the pentagon diagram: 
			%name correct?............................................................................................................
			\begin{center}
				\begin{tikzcd}
					& A \otimes (B \otimes ( C \otimes D)) 
					\arrow[ld, "\id \otimes \alpha " ]
					\arrow[rd, "\alpha " ]
					&
					\\
					A \otimes ((B \otimes C ) \otimes D)
					\arrow[d, "\alpha" ]
					&
					&
					(A \otimes B) \otimes (C \otimes D)
					\arrow[d, "\alpha" ]
					\\
					(A \otimes (B \otimes C)) \otimes D
					\arrow[rr, "\alpha \otimes \id " ]
					&
					&
					((A \otimes B) \otimes C) \otimes D
				\end{tikzcd}
			\end{center}
			
			\item the unit diagram:
			\begin{center}
				\begin{tikzcd}
					(A \otimes \Unit) \otimes B
					\arrow[rr, "\alpha " ]
					\arrow[rd, "\rho \otimes \id "' ]
					&
					&
					A \otimes (\Unit \otimes B)
					\arrow[ld, "\id \otimes \lambda " ]
					\\
					& 
					A \otimes B
					&
				\end{tikzcd}
			\end{center}
		\end{itemize}
		A \emph{symmetric monoidal category} has an additional natural isomorphism:
		\begin{align*}
			s_{A,B}: A \otimes B \rightarrow B \otimes A
		\end{align*}
		satisfying the following conditions:
		\begin{itemize}
			\item the hexagon law:
			\begin{center}
				\begin{tikzcd}
					A \otimes (B \otimes C)
					\arrow[d, "\id \otimes s_{B, C}"]
					\arrow[r, "\alpha " ]
					&
					(A \otimes B) \otimes C
					\arrow[r, "s_{A\otimes B, C} "]
					&
					C \otimes (A \otimes B)
					\arrow[d, "\alpha" ]
					\\
					A \otimes (C \otimes B)
					\arrow[r, "\alpha"' ]
					&
					(A \otimes C) \otimes B 
					\arrow[r, "s_{A, C} \otimes B"' ]
					&
					(C \otimes A) \otimes B
				\end{tikzcd}
			\end{center}
			\item the inverse law: $s_{A, B}^{-1} = s_{B, A} $
			\item the unit law: $\lambda = \rho s$
			\begin{center}
				\begin{tikzcd}
					\Unit \otimes A 
					\arrow[rr, "s_{\Unit, A} "]
					\arrow[rd, "\lambda "' ]
					&
					&
					A \otimes \Unit
					\arrow[ld, "\rho" ]
					\\
					&
					A
					&
				\end{tikzcd}
			\end{center}
		\end{itemize}
		\notimportantnote[inline]{Diagramme verschönern}
		%........................................................................................................................................
	\end{definition}
	
	\todo[inline]{fix associators throughout the document (direction)}
	
	\begin{definition}[Closed monoidal category]
		\label{def: closedMonoCat}
		\todo{lengthy, rambling... muss eleganter gehen}
		A monoidal category $\category{C} $ is called \emph{right closed}, resp. \emph{left closed}, iff there is a functor 
		$- \multimap -: \category{C}^\opp \times \category{C} \rightarrow \category{C} $
		, resp. 
		$- \multimapinv -: \category{C} \times \category{C}^\opp \rightarrow \category{C} $
		, such that there is a natural isomorphism 
		$\Hom(A \otimes B, C) \cong \Hom(B, A\multimap C)$
		, resp. 
		$\Hom(A \otimes B, C) \cong \Hom(A, C\multimapinv B)$
		, i.e. iff the functor $- \otimes B$, resp. $A \otimes - $, has a right adjoint.
		A monoidal $\category{C} $ is called \emph{biclosed} iff it is both left and right closed.
		
		If $\category{C} $ is symmetric, these functors coincide we call the category \emph{closed}.
		
		The functors 
		$- \multimap -: \category{C}^\opp \times \category{C} \rightarrow \category{C} $ 
		and 
		$- \multimapinv -: \category{C} \times \category{C}^\opp \rightarrow \category{C} $
		are called \emph{internal hom} functors.
	\end{definition}
	
	
	\begin{definition}[Enriched category]
		\label{def: enrichment}
		Let $(\category{M}, \Unit, \alpha, \lambda, \rho) $ be a monoidal category. An $\category{M} $\emph{-enriched category} $\category{C} $ consists of the following data:
		\begin{itemize}
			\item 
			A class $\Obj(\category{C}) $ of objects;
			\item 
			For every pair of objects $(A, B) \in \Obj(\category{C}) \times \Obj(\category{C}) $ an object 
			$\category{C}(A, B) \in \Obj(\category{M}) $
			called \emph{hom object};
			\item 
			For every triple of objects 
			$(A, B, C) \in \Obj(\category{C}) \times \Obj(\category{C}) \times \Obj(\category{C}) $
			a morphism 
			$\Morph(\category{M}) \ni \circ_{A, B, C}: \category{C}(B, C) \otimes \category{C}(A, B) \rightarrow \category{C}(A, C) $
			called \emph{composition morphism};
			\item 
			For every object $A \in \category{C} $ a morphism $ \id_A : \Unit \rightarrow \category{C}(A, A) $;
		\end{itemize}
		such that the following diagrams commute:
		\begin{center}
			\begin{tikzcd}
				(\category{C}(C, D) \otimes \category{C}(B, C) ) \otimes \category{C}(A, B)
				\arrow[d, " \circ_{B, C, D} \otimes \id_{\category{C}(A, B) } " ]
				\arrow[rr, "\alpha " ]
				&
				&
				\category{C}(C, D) \otimes (\category{C}(B, C) \otimes \category{C}(A, B))
				\arrow[d, "\id_{\category{C}(C, D)} \otimes  \circ_{A, B, C} " ]
				\\
				\category{C}(D, B) \otimes \category{C}(A, B)
				\arrow[dr, "\circ_{A, B, D} "' ]
				&
				&
				\category{C}(C, D) \otimes \category{C}(A, C)
				\arrow[dl, "\circ_{A, C, D} " ]
				\\
				&
				\category{C}(A, D)
				&
			\end{tikzcd}
		\end{center}
		
		\begin{center}
			\begin{tikzcd}
				\Unit \otimes \category{C}(A, B)
				\arrow[r, "\lambda" ]
				\arrow[d, "\id_B \otimes \id_{\category{C}(A,B) }"' ]
				&
				\category{C}(A, B)
				&
				\category{C}(A, B) \otimes \Unit 
				\arrow[l, "\rho "' ]
				\arrow[d, "\id_{\category{C}(A, B)} \otimes \id_A " ]
				\\
				\category{C}(B, B) \otimes \category{C}(A, B)
				\arrow[ru, "\circ_{A, B, B}"']
				&
				&
				\category{C}(A, B) \otimes \category{C}(A, A)
				\arrow[lu, "\circ_{A, A, B}"]
			\end{tikzcd}
		\end{center}
	\end{definition}
	
	\begin{definition}[Enriched functors]
		\label{def: enrichedFunctors}
		Let $\category{C} $ and $\category{D} $ be $\category{M} $-enriched categories. An \emph{enriched functor} $\functor{F} : \category{C} \rightarrow \category{D} $ consists of:
		\begin{itemize}
			\item 
			a function $\functor{F}_0 : \Obj(\category{C}) \rightarrow \Obj(\category{D}) $ between the objects
			\item 
			a family of morphisms in $\category{M} $: $\functor{F}_{A,B} : \category{C}(A, B) \rightarrow \category{D}(\functor{F}_0 A , \functor{F}_0 B) $
		\end{itemize}
		such that the following diagrams commute:
		\begin{center}
			\begin{tikzcd}
				\category{C}(B, C) \otimes \category{C}(A, B)
				\arrow[r, "\circ " ]
				\arrow[d, "\functor{F}_{B, C} \otimes \functor{F}_{A, B} " ]
				&
				\category{C}(A,C)
				\arrow[d, "\functor{F}_{A, C} " ]
				\\
				\category{D}(\functor{F}_0 B , \functor{F}_0 C) \otimes \category{D}(\functor{F}_0 A , \functor{F}_0 B)
				\arrow[r, "\circ " ]
				&
				\category{D}(\functor{F}_0 A , \functor{F}_0 C)
			\end{tikzcd}
			\begin{tikzcd}
				&
				\Unit
				\arrow[dl, "\id_A "' ]
				\arrow[dr, "\id_{\functor{F}_0 A} " ]
				&
				\\
				\category{C}(A, A)
				\arrow[rr, "\functor{F}_{A, A} " ]
				&
				&
				\category{D}(\functor{F}_0 A, \functor{F}_0 A)
			\end{tikzcd}
		\end{center}
	\end{definition}
	
	
	\subsection{LinDisCats}
	\label{sec: linDisCats}
	%LDCs
	%*-auto. Cats
	
	\begin{definition}[Linearly distributive category]
		\label{def: linDisCat}
		A \emph{linearly} \emph{distributive} \emph{category} $(\category{C}, \otimes, 1, \parr, \bot)$ is a category $\category{C}$ consisting of:
		\begin{itemize}
			\item a monoidal category $(\category{C}, \otimes, 1, \alpha_\otimes, \lambda_\otimes, \rho_\otimes)$, with $\otimes$ called "tensor";
			\item a monoidal  category $(\category{C}, \parr, \bot, \alpha_\parr, \lambda_\parr, \rho_\parr)$, with $\parr$ called "par";
			\item two natural transformations called left and right \emph{linear distributors} respectively:
			\begin{align*}
				\partial_L:& A \otimes (B \parr C) \rightarrow (A\otimes B) \parr C
				\\
				\partial_R:& (A \parr B) \otimes C \rightarrow A \parr (B \otimes C)
			\end{align*}
		\end{itemize}
		satisfying the following coherence conditions:
		\begin{itemize}
			\item coherence between the distributors and unitors:
			\begin{center}
				\begin{tikzcd}
					1 \otimes (A \parr B)
					\arrow[rr, "\partial_L " ]
					\arrow[rd, "\lambda_\otimes "' ]
					&
					&
					(1 \otimes A) \parr B
					\arrow[ld, "\lambda_\otimes \parr \id " ]
					\\
					&
					A \parr B
					&
				\end{tikzcd}
				\begin{tikzcd}
					(A \parr B) \otimes 1
					\arrow[rr, "\partial_R " ]
					\arrow[rd, " \rho_\otimes "' ]
					&
					&
					A \parr (B \otimes 1)
					\arrow[ld, "\id \otimes \rho_\otimes " ]
					\\
					&
					A \parr B
					&
				\end{tikzcd}
				
				\begin{tikzcd}
					A \otimes (B \parr \bot ) 
					\arrow[rr, "\partial_L " ]
					\arrow[dr, "\id \otimes \rho_\parr "' ]
					&
					&
					(A \otimes B) \parr \bot
					\arrow[dl, "\rho_\parr " ]
					\\
					&
					A\otimes B
					&
				\end{tikzcd}
				\begin{tikzcd}
					(\bot \parr A) \otimes B
					\arrow[rr, "\partial_R " ]
					\arrow[dr, "\lambda_\parr \otimes \id "' ]
					&
					&
					\bot \parr (A \otimes B)
					\arrow[dl, "\lambda_\parr " ]
					\\
					&
					A \otimes B
					&
				\end{tikzcd}
			\end{center}
			\formatnote{Diagramme besser alignen}
			%Diagramme besser alignen.............................................................................................................
			\item coherence between distributors and associators:
			\begin{center}
				\begin{tikzcd}
					&
					(A \otimes B) \otimes (C \parr D)
					\arrow[rd, "\partial_L " ]
					\arrow[ld, "\alpha_\otimes " ]
					&
					\\
					A \otimes (B \otimes (C \parr D))
					\arrow[d, "\id \otimes \partial_L " ]
					
					&
					&
					((A \otimes B) \otimes C) \parr D
					\arrow[d, "\alpha_\otimes \parr \id " ]
					\\
					A \otimes ((B \otimes C) \parr D)
					\arrow[rr, "\partial_L " ]
					&
					&
					(A \otimes (B \otimes C)) \parr D
				\end{tikzcd}
				
				\begin{tikzcd}
					&
					A \otimes (B \parr (C \parr D))
					\arrow[ld, "\id \otimes \alpha_\parr " ]
					\arrow[rd, "\partial_L " ]
					&
					\\
					A \otimes ((B \parr C) \parr D)
					\arrow[d, "\partial_L " ]
					&
					&
					(A \otimes B) \parr (C \parr D)
					\arrow[d, "\alpha_\parr " ]
					\\
					(A \otimes (B \parr C)) \parr D
					\arrow[rr, "\partial_L \parr \id " ]
					&
					&
					((A \otimes B) \parr C) \parr D    \end{tikzcd}
				
				\begin{tikzcd}
					&
					(A \parr B) \otimes (C \otimes D)
					\arrow[ld, "\alpha_\otimes " ]
					\arrow[rd, "\partial_R " ]
					&
					\\
					((A \parr B) \otimes C) \otimes D
					\arrow[d, "\partial_R \otimes \id " ]
					&
					&
					A \parr (B \otimes (C \otimes D))
					\arrow[d, "\id \parr \alpha_\otimes " ]
					\\
					(A \parr (B \otimes C)) \otimes D
					\arrow[rr, "\partial_R " ]
					&
					&
					A \parr ((B \otimes C) \otimes D)
				\end{tikzcd}
				
				\begin{tikzcd}
					&
					((A \parr B) \parr C) \otimes D
					\arrow[ld, "\alpha_\parr \otimes \id " ]
					\arrow[rd, "\partial_R " ]
					&
					\\
					(A \parr (B \parr C)) \otimes D
					\arrow[d, "\partial_R " ]
					&
					&
					(A \parr B) \parr (C \otimes D)
					\arrow[d, "\alpha_\parr " ]
					\\
					A \parr ((B \parr C) \otimes D)
					\arrow[rr, "\id \parr \partial_R " ]
					&
					&
					A \parr (B \parr (C \otimes D))
				\end{tikzcd}
			\end{center}
			\item coherence between the distributors:
			\begin{center}
				\begin{tikzcd}
					&
					(A \parr B) \otimes (C \parr D)
					\arrow[rd, "\partial_L " ]
					\arrow[ld, "\partial_R " ]
					&
					\\
					((A \parr B) \otimes C) \parr D
					\arrow[d, "\partial_R \parr \id " ]
					&
					&
					A \parr (B \otimes (C \parr D))
					\arrow[d, "\id \parr \partial_L " ]
					\\
					(A \parr (B \otimes C)) \parr D
					&
					&
					A \parr ((B \otimes C) \parr D)
					\arrow[ll, "\alpha_\parr " ]
				\end{tikzcd}
				\formatnote[inline]{obiges Diagramm rumdrehen}
				%.......................................................................................................
				\formatnote[inline]{\textbackslash parr \textbackslash id sieht kacke aus...}    %..................................................................................
				\begin{tikzcd}
					&
					A \otimes ((B \parr C) \otimes D)
					\arrow[ld, "\alpha_\otimes " ]
					\arrow[rd, "\id \otimes \partial_R " ]
					&
					\\
					(A \otimes (B \parr C)) \otimes D
					\arrow[d, "\partial_L \otimes \id " ]
					&
					&
					A \otimes (B \parr (C \otimes D))
					\arrow[d, "\partial_L " ]
					\\
					((A \otimes B) \parr C) \otimes D
					\arrow[rr, "\partial_R " ]
					&
					&
					(A \otimes B) \parr (C \otimes D)
				\end{tikzcd}
			\end{center}
		\end{itemize}
		\formatnote[inline]{itemize anpassen?}
		%.............................................................................
	\end{definition}
	
	
	\subsection{*-aut. cats}
	\label{sec: autoCats}
	\todo[inline]{structure an Mellies, CatSemOfLinLog, anpassen; equivalence Srinivasan and Barr definitions}
	
	\begin{definition}[Dual object]
		\label{def: dualObj}
		%source: srinivasan
		Let $\category{C} $ be a linearly distributive category and $A, B \in \Obj(\category{C}) $, then we call $B $ \emph{left dual} (or \emph{left linearly adjoint}) to $A$, if there are morphisms $\eta: 1 \rightarrow B \parr A$ and $\varepsilon: A \otimes B \rightarrow \bot $, called \emph{unit} and \emph{counit} resp., such that the following diagrams commute:
		\begin{center}
			\begin{tikzcd}
				B
				\arrow[r, equal, "\id" ]
				&
				B
				\arrow[d, "l_\otimes^{-1} " ]
				\\
				B \parr \bot
				\arrow[u, "r_\parr " ]
				&
				1 \otimes B
				\arrow[d, "\eta \otimes \id " ]
				\\
				B\parr (A \otimes B)
				\arrow[u, "\id \parr \varepsilon " ]
				&
				(B \parr A) \otimes B
				\arrow[l, "\partial_R " ]
			\end{tikzcd}
			\begin{tikzcd}
				A
				\arrow[r, equal, "\id " ]
				&
				A
				\arrow[d, "r_\otimes^{-1} " ]
				\\
				\bot \parr A
				\arrow[u, "l_\parr " ]
				&
				A \otimes 1
				\arrow[d, "\id \otimes \eta " ]
				\\
				B\parr (A \otimes B)
				\arrow[u, "\varepsilon \parr \id " ]
				&
				(B \parr A) \otimes B
				\arrow[l, "\partial_L " ]
			\end{tikzcd}
			\formatnote[inline]{das geht schöner}
		\end{center}
		
		We write $(\eta, \varepsilon): B \dashV A $ and also call $A$ \emph{right dual} to $B$. 
	\end{definition}
	
	
	\begin{lemma}
		\label{lemma: isoLinDisCat}
		{ }
		\begin{enumerate}
			\item In an LDC: if 
			$B \dashV A $ and $C \dashV A $
			, then $B$ and $C$ are isomorphic
			\item In a symmetric LDC: 
			$(\eta, \varepsilon): B \dashV A 
			\Longleftrightarrow 
			(\eta s_{\parr}, s_{\otimes} \varepsilon): A \dashV B $ 
		\end{enumerate}
	\end{lemma}
	
	\todo[inline]{proof }
	%(BCS: "Feedback for linearly distributive categories: traces and fixpoints)
	%comment..............................................................................................................................
	
	\todo[inline]{Srinivasan Lemma 2.9 (iii)?}
	\todo[inline]{definition (iso)mix cat?}
	%comment.............................................................................................................................
	
	\begin{definition}[*-autonomous categories (Srinivasan)]
		\label{def: autoCatSrinivasan}
		\notimportantnote{is Srinivasan the origin of this version?}
		An LDC $\category{C} $ in which for every Object $A \in \category{C} $ there exists a left and right dual, resp. $(\eta*, \varepsilon*): A^* \dashV A $ and $(*\eta, *\varepsilon): A \dashV \leftindex^* {A} $, is called \emph{*-autonomous category}.
	\end{definition}
	
	
	
	\begin{definition}[*-autonomous categories (Barr (95), A)]
		\label{def: autoCatBarrA}
		\notimportantnote{find a way to make letter part of definition numbering?}
		A *-autonomous category is a biclosed monoidal category $\category{C} $ with a closed functor $(-)^*: \category{C} \rightarrow \category{C}^\opp $, which is a strong equivalence of categories.     
	\end{definition}
	%Barr, 95
	\todo[inline]{necessary? meaning of "closed" (functor between C-enriched categories)}
	
	\begin{definition}[Dualizing object]
		\label{def: dualizingObj}
		Let $\category{C} $ be a biclosed monoidal category. 
		\todo{details (Barr, 99)}
		An object $\bot $ is called a dualizing object if for every $A \in \Obj(\category{C}) $ the natural map $A \rightarrow \bot \multimapinv (A \multimap \bot) $ gotten by transposing $\id: A \multimap \bot \rightarrow A\multimap \bot $ twice is an isomorphism. 
	\end{definition}
	
	\begin{definition}[*-autonomous categories (Barr, B)]
		\label{def: autoCatBarrB}
		A *-autonomous category is a biclosed category with a dualizing object.
	\end{definition}
	\todo{main definition, Barr}
	
	
	\begin{definition}[*-autonomous categories (Barr, C)]
		\label{def: autoCatBarrC}
		A *-autonomous category is a monoidal category $\category{C} $ equipped with an equivalence $(-)^* : \category{C} \rightarrow \category{C}^\opp $ such that there is a natural isomorphism 
		\begin{align*}
			\Hom(A, B^*) \rightarrow \Hom(1, (A \oplus B)^*)
		\end{align*}
	\end{definition}
	
	
	\begin{definition}[*-autonomous categories (Barr, D)]
		\label{def: autoCatBarrD}
		A *-autonomous category is a closed category $\category{C} $ in the sense of 
		\todo{nachschauen}
		Eilenberg and Kelly (1966) together with an equivalence $(-)^* : \category{C}^\opp \rightarrow \category{C} $ such that 
		\begin{align*}
			A \multimap (B \multimapinv C) \cong (A \multimap B) \multimapinv C
		\end{align*}
		where $A \multimapinv B := A^* \multimap B^* $.
	\end{definition}
	\questionnote[inline]{unnötig? C impliziert offensichtlich D, von D nach C sind drei Zeilen.}
	
	
	\todo[inline]{Äquivalenz von Barrs Definitionen:}
	
	\begin{theorem}
		\label{thrm: BarrEqui}
		Barr's definitions are equivalent.
	\end{theorem}
	
	\todo[inline]{Folgerung Barr A $\Rightarrow$ Barr B} 
	
	\begin{proof}
		\formatnote{define useful control sequence}
%		\implication{Barr A}{Barr B}
		Suppose the category $\category{C} $ and the functor $(-)^\bot: \category{C} \rightarrow \category{C}^\opp $ fulfill the conditions of 
		Definition \ref{def: autoCatBarrA} (Barr A)
	\end{proof}
	
	\todo[inline]{Folgerung Barr B $\Rightarrow$ Barr c}
	\todo[inline]{Folgerung Barr C $\Rightarrow$ Barr A}
	
	\questionnote[inline]{what did I need Barr's definition for, again?}
	
	
	\todo[inline]{Cats as LinLog semantics and LinLog syntax as cat}
	%Cats as LinLog semantics
	
	\todo[inline]{bibliography }
	
	\printbibliography
	
	\newpage
	\listoftodos
	
\end{document}
