\documentclass[DIN, pagenumber=false, fontsize=11pt, parskip=half, colorinlistoftodos, svgnames]{scrartcl}
%\\
%\usepackage[utf8]{inputenc}
%\usepackage[ngerman]{babel}
\usepackage[T1]{fontenc}

\usepackage{xeCJK} %why am I doing this...?
\xeCJKsetup{CJKmath=true}
\setCJKmainfont{Noto Serif JP} 
\setCJKsansfont{Noto Sans CJK JP Light}
\setCJKmonofont{Noto Sans Mono CJK JP}


%\usepackage{cite}
\usepackage{enumitem}
%\usepackage{hyperref} %Das lädt automatisch...?

\usepackage[backend=biber, style=alphabetic ]{biblatex}
\addbibresource{bib/quellen.bib}

\usepackage{nchairx} %Lehrstuhl package
\usepackage{mathtools}			
\usepackage{amssymb}
%\usepackage{amsthm} %conlfict with nChairX
\usepackage{mathabx} %for dashV etc. conflict with amssymb
\usepackage{eurosym}
\usepackage{leftindex} 
\usepackage{graphicx}
\usepackage{pgf}
\usepackage{tikz}
\usetikzlibrary{cd}
\usepackage{typearea}
\usepackage[]{scrlayer-scrpage} %scrpage2 is obsolete and not supported by MikTex and TeX live 
\usepackage{subcaption}
\usepackage{lastpage}
\usepackage{multicol} %??
\usepackage{float}
\usepackage[hidelinks]{hyperref}
%\usepackage[table]{xcolor} %Übernommen aus Protokollen letzten Jahres, Nutzen unbekannt:
%\usepackage[margin=10pt,font=small,labelfont=bf]{caption}
\usepackage{textgreek}
\usepackage{marvosym} %Eurozeichen. mehr?

%LinLog (par and with, etc.)
\usepackage{cmll}

%Logic
\usepackage{bussproofs}

%Tables
\usepackage{tabularx}
\newcolumntype{x}{>{\centering\arraybackslash}X}

\usepackage{graphicx}
\usepackage{tikz}

%scalable prooftrees:
\newenvironment{scprooftree}[1]%
{\gdef\scalefactor{#1}\begin{center}\proofSkipAmount \leavevmode}%
	{\scalebox{\scalefactor}{\DisplayProof}\proofSkipAmount \end{center} }

\usepackage[]{todonotes} %to do notes...
%package options: obeyDraft , obeyFinal
\setuptodonotes{tickmarkheight=0.1cm}
%different types of todonotes:
\newcommand{\formatnote}[2][]{\todo[color=cyan!40, #1]{#2}}
\newcommand{\questionnote}[2][]{\todo[color=purple!80, #1]{#2}}
\newcommand{\urgentnote}[2][]{\todo[color=red, #1]{#2}}
\newcommand{\notimportantnote}[2][]{\todo[color=LightPink, #1]{#2}}



\usepackage[a4paper,margin=3cm]{geometry}

%definition, theorem, remark etc. (already in nchairX)
%\theoremstyle{definition}
%\newtheorem{definition}{Definition}[section]



%\addto\captionsngerman{\renewcommand\figurename{Abb.}}
%\addto\captionsngerman{\renewcommand\tablename{Tab.}}

%\usepackage{siunitx} %SI-Einheiten. clash with nchairx (\unit)
%\DeclareSIUnit[]{\euro}{\text{\EUR}}

%Nützliche Shortcuts:
\def\SP#1{\textsuperscript{{#1}}} % Index oben, textmode
\def\SB#1{\textsubscript{{#1}}} % Index unten, textmode
\def\SPSB#1#2{\rlap{\SP{#1}}\SB{#2}} % Kombi Index oben und unten, textmode
\def\QR#1#2{\raisebox{1ex}{\ensuremath{#1}}\ensuremath{\mkern-3mu}\big/\ensuremath{\mkern-3mu}\raisebox{-1ex}{\ensuremath{#2}}} % schräger Bruch
\newcommand{\dd}[1]{\mathrm{d}#1} % kleines d Diff.op
\def\Ree{\operatorname{Re}} % real part Re (not fraktur R)
\def\Imm{\operatorname{Im}} % imaginary part Im (not fraktur I)
%\DeclareMathOperator{\tr}{tr} % Träger (already in nChairX)

%some stuff für Beweise mit mehreren Teilen:
\newcommand{\implication}[2]{%
	\enquote{\mbox{$\text{#1}\implies\text{#2}$}}%
	\enspace\ignorespaces
}
\newcommand{\statement}[1]{%
	#1\enspace\ignorespaces
}


\usepackage{csvsimple} %Import von csv-Dateien


%better vdots:
\DeclareRobustCommand{\svdots}{% s for `scaling'
	\vbox{%
		\baselineskip=0.33333\normalbaselineskip
		\lineskiplimit=0pt
		\hbox{.}\hbox{.}\hbox{.}%
		\kern-0.2\baselineskip
	}%
}


%==============================================================================


\title{LinLog und LinDisCats}
\author{Max Wiegand}

%\sisetup{output-decimal-marker = {,}, per-mode = fraction, separate-uncertainty, exponent-product = \cdot,}
%\renewcommand{\thesection}{}
%\renewcommand\thesubsection{\Alph{subsection}.)}
\allowdisplaybreaks

%=============================================================================
%text starts 

\begin{document}
	\maketitle
	
	\section{Introduction}
	
%	\subsection{Intuition}
%	\label{sec: intuition}
	
	%................................
	%\todo[inline, color=green!50]{feels kinda rambling... combine intuition with syntax definition below}
	%comment..............................................................................
	
	Classical (and intuitionistic) logic deals with the propagation of stable truth values. If one has a true sentence $A$ and an an implication $A \Rightarrow B$, then $B$ follows while $A$ remains true. However, real-life implications are often causal and modify their premises. They cannot therefore be iterated arbitrarily. For example if $A$ describes the ownership of 1\euro\ and $B$ owning a chocolate bar, an implication $A \multimap B$ (to be formally introduced later) would describe the process of buying such a chocolate bar for 1\euro, losing the 1\euro\ in the process.
	
	While such dealings with resources can of course be modeled in classical logic, it is easier in the resource-sensitive \emph{linear logic}, first described by Girard in 1987 \cite{girard87}. 
	
	Here, we have two conjunctions simultaneously $\otimes$ ("times" or "tensor") and $\with$ ("with"), which describe the availability of resources:
	
	Suppose $C$ is the ownership of a cookie and it costs also 1\euro\ (i.e. we have $A\multimap C$). Then $B \otimes C$ states that one owns both a chocolate bar and a cookie. The implication $A \multimap B \otimes C$ is not possible, as it would mean that you are buying both, cookie and chocolate bar at the same time, for just 1\euro\ total. However, from $A \multimap B$ and $A\multimap C$ we get $A \otimes A \multimap B \otimes C$, i.e. the process of buying both for 2\euro.
	
	On the other hand, $B \with C$ states that one has a choice between either $B$ or $C$ (imagine a token). From the implications $A\multimap B$ and $A\multimap C$ we get the implication $A \multimap B \with C$, i.e. the process of buying a token to be exchanged for a chocolate bar or a cookie at a later time (with the choice lying with oneself). While this may  seem like a disjunction, both implications $B\with C \multimap B$ and $B \with C \multimap C$ (exchanging the token for either product) are provable from $B \with C$, although not simultaneously. 
	
	Dually, we have two disjunctions $\parr$ ("par") and $\oplus$ ("plus"):
	
	Suppose now that $B$ and $C$ are the ownership of a figurine of Pikachu or Mew respectively. Then $B \oplus C$ may be the ownership of a Kinder Egg containing either figurine. This means when buying that egg ($A \multimap B \oplus C$) we do not know which one we will get. 
	
	Our second disjunction, dual to $\otimes$, can be understood by linear implication and the linear negation (denoted as $(\cdot)^\bot$): Under the interpretation of ownership the linear negation is no interpreted as the absence of ownership but as negative ownership, i.e. debt. That means the negation of owning 1\euro, $A$, is owing someone 1\euro, $A^\bot$. With the par operator we can now write the linear implication $A \multimap B$ symmetrically as $A^\bot \parr B$. 
	%\todo[inline]{(explicit interpretation?! shared pool between people containing Pika and Mew -> you have to get rid of one to use the other?)}
	%Interpretation von par...............................................................................................................................
	%comment...............................................................................................................................................
	
	In order to regain our stable truths known from classical logic, we need to employ two unitary connectives $\oc$ ("of course" or "bang") and $\wn$ ("why not"). 
	The bang operator informs us that there is an possibly infinite amount of a resource: 
	The statement $\oc A$ translates into the ownership of an amount of money that is large enough for us to ignore resource sensitivity. 
	Imagine for example a billionaire buying a Pokemon figurine: His amount of money will not be noticeably smaller after buying the figurine. We can informally say $\oc A = (1 \with A) \otimes (1 \with A) \otimes \cdots $ and therefore view classical and intuitionistic logic as some sort of limit of linear logic, just as classical mechanics is a limit of quantum mechanics and the theory of relativity. 
	
	%......................................................
	%\todo[inline]{interpretation of ?wn and constants.}
	%comment........................................................................................................................................
	
	
	Various fragments of linear logic can be modeled by monoidal categories with additional structures. We will see this with the multiplicative fragment and linearly distributive categories.
	
%	\urgentnote[inline]{mono cat as model for various fragments}
%	\urgentnote[inline]{applications}
	
	
	
	
	

	%---------------------------------------------------------------------------------------------------------------------------------------------------
	
	%___________________________________________________________________________________________________________________________________________________________________________________________________________________________________________________________________________________________
	\section{Categorical Preliminaries}
	
	In order to model linear logic with certain categories, we require a short introduction into category theory. 
	We will first define the fundamental concepts such as \emph{categories}, \emph{functors} and \emph{natural} \emph{transformations}. 
	Using these, we then introduce \emph{monoidal} \emph{categories}, categories with a tensor product, which form the foundation for any categorical model of linear logic. 
	As we ultimately want to model multiplicative linear logic, we arrive at the definition of \emph{linearly} \emph{distributive} \emph{categories}, first introduced by Cocket and Seely \cite{cockett&seely97} as \emph{weakly} \emph{distributive} \emph{categories}. 
	Finally, we will take a quick look at \emph{*-autonomous} \emph{categories}, which form a model of multiplicative linear logic with linear negation \notimportantnote{", and their relation to LDCs"}.
	
	\todo[inline]{Einleitung geht besser}
	
	\subsection{Categories}
	\label{sec: catSec}
	%Tensorcats?
	
	\begin{definition}[Category]
		\label{def: cat}
		A Category $\category{C}$ consists of the following data: 
		\begin{itemize}
			\item 
				A class $\Obj(\category{C})$ of \emph{objects}.
			\item 
				For every pair of objects $A, B \in \Obj(\category{C})$ there is a class $\Hom(A, B)$ of \emph{morphisms} $f: A \rightarrow B$ from $A$ to $B$. 
				We denote the class of all morphisms of $\category{C}$ with $\Morph{(\category{C})}$
			\item 
				Morphisms \emph{compose}: For $f \in \Hom(A, B)$ and $g \in \Hom(B, C)$ there is a morphism $g \circ  f \in \Hom(A, C)$. 
				That composition is associative: 
				\begin{align*}
					h \circ (g \circ f) = (h \circ g) \circ f
				\end{align*}
				We will write $gf$ for $g\circ f$ when appropriate.
			\item 
				For every object $A$ there is an \emph{identity morphism} $\id_A \in \Hom(A, A)$:
				\begin{align*}
					f \circ \id_A = f, \quad \id_B \circ f = f
				\end{align*}
				for $f \in \Hom(A, B)$
		\end{itemize}
		If the hom-classes are sets, we call the category \emph{locally small}. If the object class is a set, we call the category \emph{small}. Otherwise, we call the category \emph{large}.
		
		If we have a category $\category{C} $, we call its \emph{opposite} category $\category{C}^\opp $ the category with the following structure:
		\begin{itemize}
			\item $\Obj{\left(\category{C}^\opp \right)} = \Obj(\category{C}) $
			\item $\forall A, B \in \Obj(\category{C}): 
			\Hom_{\category{C}^\opp }(A,B) = \Hom_{\category{C}}(B,A)$
		\end{itemize}
	\end{definition}
	
	We are often interested in the relationship between categories and how their structures can be translated into one another. Functors serve this purpose:
	
	\begin{definition}[Functor]
		\label{def: functor}
		A \emph{(covariant) functor} $\functor{F}: \category{C} \rightarrow \category{D} $ between two categories $\category{C}$ and $\category{D} $ is a function mapping each object $A \in  \Obj(\category{C}) $ to an object $F(A) \in \Obj(D) $ and each morphism $ f \in \Hom(A,B) $ to a morphism $\functor{F}(f) \in \Hom(\functor{F}(A), \functor{F}(B))$ such that identity and composition are preserved:
		\begin{align*}
			\functor{F}(\id_A) = \id_{\functor{F}(A)},
			\quad
			\functor{F}(g\circ f) = \functor{F}(g) \circ \functor{F}(f)
		\end{align*}
		
		A functor $\functor{F}: \category{C}^\opp \rightarrow \category{D} $ is called \emph{contravariant} on $\category{C} $.
		
		We will drop the parentheses when appropriate.
	\end{definition}
	
	\todo[inline]{Übergangssatz}
	
	\begin{definition}[Natural transformation]
		\label{def: natTrans}
		A \emph{natural transformation} $\tau: \functor{F} \rightarrow \functor{G} $ between two functors $\functor{F}, \functor{G}: \category{C} \rightarrow \category{D} $ is family of morphisms in $\category{D}$:
		\begin{align*}
			\tau = \{ \tau_A : \functor{F}A \rightarrow \functor{G}A \,| A \in \Obj(\category{C})\}
		\end{align*}
		such that $\tau_B\functor{F}(f) = \functor{F}(f)\tau_A $ for all $f: A \rightarrow B \in \Morph(\category{C}) $, i.e. the following diagram commutes:
		
		\begin{center}
			\begin{tikzcd}
				\functor{F} A 
				\arrow[r, "\functor{F}f " ]
				\arrow[d, "\tau_A" ] 
				&
				\functor{F} B
				\arrow[d, "\tau_B" ]
				\\
				\functor{G} A 
				\arrow[r, "\functor{G}f" ]
				&
				\functor{G} B
			\end{tikzcd}
		\end{center}
		
		If $\tau_A$ is an isomorphism for all $A \in \Obj(\category{C}) $, we call $\tau$ a \emph{natural} \emph{isomorphism}.
		We will often represent a natural transformation by a single one of its members and also drop the index when appropriate, i.e. denoting 
		$\tau: \functor{F} \rightarrow \functor{G} $ 
		by 
		$\tau : \functor{F}A \rightarrow \functor{G}A $ .
	\end{definition}
	
	
	%\todo[inline]{definition: equivalence of categories}
	%definition only relevant for Barr's definitions of *-auto. cats
	%comment.....................................................................................................................................

	
	\begin{definition}[Adjunction]
		\label{def: adjunction}
%		\notimportantnote[]{source: Brandenburg}
%		comment.....................................................................................................
		Let $\category{C}$ and $\category{D} $ be categories. We call $\functor{F}: \category{C} \rightarrow \category{D} $ the \emph{left adjoint} of $\functor{G}: \category{D} \rightarrow \category{C} $ and $\functor{G} $ the \emph{right adjoint} of $\functor{F} $ if there is an isomorphism that is natural in $A \in \category{C} $ and $B \in \category{D} $:
		\begin{align*}
			\forall A \in \category{C}, \forall B \in \category{D}:
			\Hom_\category{D}(\functor{F}A, B) 
			\cong \Hom_\category{C}(A, \functor{G}B)
		\end{align*}
		Equivalently, we call $\functor{F} $ the left adjoint of $\functor{G} $ if there are natural transformations 
		\begin{align*}
			\eta &: \id_\category{C} \rightarrow \functor{G} \circ \functor{F}
			\\
			\varepsilon &: \functor{F} \circ \functor{G} \rightarrow \id_\category{D}
		\end{align*}
		fulfilling the following condition: 
		
		\begin{align*}
			\id_{\functor{F}A} 
			&= \varepsilon_{\functor{F}A} \circ \functor{F}(\eta_A)
			\\
			\id_{\functor{G}B} 
			&= \functor{G}(\varepsilon_B) \circ \eta_{\functor{G}B}
		\end{align*}
		
		%cut content.....................................................................................................
		\iffalse
		i.e. the following diagrams commute:
		\begin{center}
			\begin{tikzcd}
				&
				\functor{F} \circ \functor{G} \circ \functor{F}
				\arrow[rd, "\varepsilon \bullet \functor{F} " ]
				&
				\\
				\functor{F}\id_\category{C}
				\arrow[ru, "\functor{F} \bullet \eta " ]
				\arrow[rr, equal, "\id" ]
				&
				&
				\id_\category{C} \circ \functor{F}
			\end{tikzcd}
			\begin{tikzcd}
				&
				\functor{G} \circ \functor{F} \circ \functor{G}
				\arrow[rd, "\functor{G} \bullet \varepsilon " ]
				&
				\\
				\id_\category{C} \circ \functor{G}
				\arrow[ru, "\eta \bullet \functor{G} "]\
				\arrow[rr, equal, "\id " ]
				&
				&
				\functor{G} \circ \id_\category{D}
			\end{tikzcd}
		\end{center}
		\todo[inline]{cut triangle diagrams or define $\bullet $ } %comment......................................................................................................................................
		\fi
		%end of cut content.............................................................................................
		
		We write $F \dashv G$.
	\end{definition}
	
	
	\begin{definition}[Monoidal category]
		\label{def: monoCat}
		A \emph{monoidal category} $(\category{C}, \otimes, \Unit, \alpha, \lambda, \rho)$ is a category $\category{C}$ with the following additional data:
		\begin{itemize}
			\item 
				a functor $\otimes : \category{C} \times \category{C} \rightarrow \category{C} $, called \emph{monoidal} or \emph{tensor product};
			\item 
				an object $\Unit \in  \Obj(\category{C})$, called \emph{unit object};
			\item 
				a natural isomorphism $\alpha: A \otimes (B \otimes C ) \rightarrow (A \otimes B) \otimes C$, called \emph{associator};
			\item 
				a natural isomorphism $\lambda: \Unit \otimes A \rightarrow A$, called left unitor;
			\item 
				a natural isomorphism $\rho: A \otimes \Unit \rightarrow A$, called right unitor;
		\end{itemize}
		such that the following diagrams commute:
		\begin{itemize}
			\item the pentagon diagram: 
			%name correct?............................................................................................................
			\begin{center}
				\begin{tikzcd}
					& A \otimes (B \otimes ( C \otimes D)) 
					\arrow[ld, "\id \otimes \alpha " ]
					\arrow[rd, "\alpha " ]
					&
					\\
					A \otimes ((B \otimes C ) \otimes D)
					\arrow[d, "\alpha" ]
					&
					&
					(A \otimes B) \otimes (C \otimes D)
					\arrow[d, "\alpha" ]
					\\
					(A \otimes (B \otimes C)) \otimes D
					\arrow[rr, "\alpha \otimes \id " ]
					&
					&
					((A \otimes B) \otimes C) \otimes D
				\end{tikzcd}
			\end{center}
			
			\item the unit diagram:
			\begin{center}
				\begin{tikzcd}
					(A \otimes \Unit) \otimes B
					\arrow[rr, "\alpha " ]
					\arrow[rd, "\rho \otimes \id "' ]
					&
					&
					A \otimes (\Unit \otimes B)
					\arrow[ld, "\id \otimes \lambda " ]
					\\
					& 
					A \otimes B
					&
				\end{tikzcd}
			\end{center}
		\end{itemize}
		A \emph{symmetric monoidal category} has an additional natural isomorphism:
		\begin{align*}
			s_{A,B}: A \otimes B \rightarrow B \otimes A
		\end{align*}
		satisfying the following conditions:
		\begin{itemize}
			\item 
				the hexagon law:
				\begin{center}
					\begin{tikzcd}
						A \otimes (B \otimes C)
						\arrow[d, "\id \otimes s_{B, C}"]
						\arrow[r, "\alpha " ]
						&
						(A \otimes B) \otimes C
						\arrow[r, "s_{A\otimes B, C} "]
						&
						C \otimes (A \otimes B)
						\arrow[d, "\alpha" ]
						\\
						A \otimes (C \otimes B)
						\arrow[r, "\alpha"' ]
						&
						(A \otimes C) \otimes B 
						\arrow[r, "s_{A, C} \otimes B"' ]
						&
						(C \otimes A) \otimes B
					\end{tikzcd}
				\end{center}
			\item 
				the inverse law: $s_{A, B}^{-1} = s_{B, A} $
			\item 
				the unit law: $\lambda = \rho s$
				\begin{center}
					\begin{tikzcd}
						\Unit \otimes A 
						\arrow[rr, "s_{\Unit, A} "]
						\arrow[rd, "\lambda "' ]
						&
						&
						A \otimes \Unit
						\arrow[ld, "\rho" ]
						\\
						&
						A
						&
					\end{tikzcd}
				\end{center}
		\end{itemize}
		%............................................................................................
		%\notimportantnote[inline]{Diagramme verschönern}
		%........................................................................................................................................
	\end{definition}
	
	
	%cut conten.................................................................................................................................................
	%cut because only needed in Barr's definitions
	\iffalse
	\begin{definition}[Closed monoidal category]
		\label{def: closedMonoCat}
		\todo{lengthy, rambling... maybe cut into three definitions?}
		A monoidal category $\category{C} $ is called \emph{right closed}, resp. \emph{left closed}, iff there is a functor 
		$- \multimap -: \category{C}^\opp \times \category{C} \rightarrow \category{C} $, 
		resp. 
		$- \multimapinv -: \category{C} \times \category{C}^\opp \rightarrow \category{C} $, 
		such that there is a natural isomorphism 
		$\Hom(A \otimes B, C) \cong \Hom(B, A\multimap C)$, 
		resp. 
		$\Hom(A \otimes B, C) \cong \Hom(A, C\multimapinv B)$,
		i.e. iff the functor $- \otimes B$, resp. $A \otimes - $, has a right adjoint.
		A monoidal $\category{C} $ is called \emph{biclosed} iff it is both left and right closed.
		
		\todo[inline]{definitely  mixed up the right and left homs there}
		
		If $\category{C} $ is symmetric, these functors coincide and we call the category \emph{closed}.
		
		The functors 
		$- \multimap -: \category{C}^\opp \times \category{C} \rightarrow \category{C} $ 
		and 
		$- \multimapinv -: \category{C} \times \category{C}^\opp \rightarrow \category{C} $
		are called \emph{internal hom} functors.
	\end{definition}
	\fi 
	
\iffalse	
	\todo[inline]{following definitions prolly unnötig:}
	
	\begin{definition}[Enriched category]
		\label{def: enrichment}
		Let $(\category{M}, \Unit, \alpha, \lambda, \rho) $ be a monoidal category. An $\category{M} $\emph{-enriched category} $\category{C} $ consists of the following data:
		\begin{itemize}
			\item 
			A class $\Obj(\category{C}) $ of objects;
			\item 
			For every pair of objects $(A, B) \in \Obj(\category{C}) \times \Obj(\category{C}) $ an object 
			$\category{C}(A, B) \in \Obj(\category{M}) $
			called \emph{hom object};
			\item 
			For every triple of objects 
			$(A, B, C) \in \Obj(\category{C}) \times \Obj(\category{C}) \times \Obj(\category{C}) $
			a morphism 
			$\Morph(\category{M}) \ni \circ_{A, B, C}: \category{C}(B, C) \otimes \category{C}(A, B) \rightarrow \category{C}(A, C) $
			called \emph{composition morphism};
			\item 
			For every object $A \in \category{C} $ a morphism $ \id_A : \Unit \rightarrow \category{C}(A, A) $;
		\end{itemize}
		such that the following diagrams commute:
		\begin{center}
			\begin{tikzcd}
				(\category{C}(C, D) \otimes \category{C}(B, C) ) \otimes \category{C}(A, B)
				\arrow[d, " \circ_{B, C, D} \otimes \id_{\category{C}(A, B) } " ]
				\arrow[rr, "\alpha " ]
				&
				&
				\category{C}(C, D) \otimes (\category{C}(B, C) \otimes \category{C}(A, B))
				\arrow[d, "\id_{\category{C}(C, D)} \otimes  \circ_{A, B, C} " ]
				\\
				\category{C}(D, B) \otimes \category{C}(A, B)
				\arrow[dr, "\circ_{A, B, D} "' ]
				&
				&
				\category{C}(C, D) \otimes \category{C}(A, C)
				\arrow[dl, "\circ_{A, C, D} " ]
				\\
				&
				\category{C}(A, D)
				&
			\end{tikzcd}
		\end{center}
		
		\begin{center}
			\begin{tikzcd}
				\Unit \otimes \category{C}(A, B)
				\arrow[r, "\lambda" ]
				\arrow[d, "\id_B \otimes \id_{\category{C}(A,B) }"' ]
				&
				\category{C}(A, B)
				&
				\category{C}(A, B) \otimes \Unit 
				\arrow[l, "\rho "' ]
				\arrow[d, "\id_{\category{C}(A, B)} \otimes \id_A " ]
				\\
				\category{C}(B, B) \otimes \category{C}(A, B)
				\arrow[ru, "\circ_{A, B, B}"']
				&
				&
				\category{C}(A, B) \otimes \category{C}(A, A)
				\arrow[lu, "\circ_{A, A, B}"]
			\end{tikzcd}
		\end{center}
	\end{definition}
	
	\begin{definition}[Enriched functors]
		\label{def: enrichedFunctors}
		Let $\category{C} $ and $\category{D} $ be $\category{M} $-enriched categories. An \emph{enriched functor} $\functor{F} : \category{C} \rightarrow \category{D} $ consists of:
		\begin{itemize}
			\item 
			a function $\functor{F}_0 : \Obj(\category{C}) \rightarrow \Obj(\category{D}) $ between the objects
			\item 
			a family of morphisms in $\category{M} $: $\functor{F}_{A,B} : \category{C}(A, B) \rightarrow \category{D}(\functor{F}_0 A , \functor{F}_0 B) $
		\end{itemize}
		such that the following diagrams commute:
		\begin{center}
			\begin{tikzcd}
				\category{C}(B, C) \otimes \category{C}(A, B)
				\arrow[r, "\circ " ]
				\arrow[d, "\functor{F}_{B, C} \otimes \functor{F}_{A, B} " ]
				&
				\category{C}(A,C)
				\arrow[d, "\functor{F}_{A, C} " ]
				\\
				\category{D}(\functor{F}_0 B , \functor{F}_0 C) \otimes \category{D}(\functor{F}_0 A , \functor{F}_0 B)
				\arrow[r, "\circ " ]
				&
				\category{D}(\functor{F}_0 A , \functor{F}_0 C)
			\end{tikzcd}
			
			\begin{tikzcd}
				&
				\Unit
				\arrow[dl, "\id_A "' ]
				\arrow[dr, "\id_{\functor{F}_0 A} " ]
				&
				\\
				\category{C}(A, A)
				\arrow[rr, "\functor{F}_{A, A} " ]
				&
				&
				\category{D}(\functor{F}_0 A, \functor{F}_0 A)
			\end{tikzcd}
		\end{center}
	\end{definition}
\fi

	
%---------------------------------------------------------------------------------------------------------------------------------------------------------------------------------------------------------------------------------------------------------------------------------------------------	
	\subsection{Linearly distributive categories }
	\label{sec: linDisCats}
	%LDCs
	%*-auto. Cats
	
	\begin{definition}[Linearly distributive category \cite{srinavsan-thesis}]
		\label{def: linDisCat}
		A \emph{linearly} \emph{distributive} \emph{category} $(\category{C}, \otimes, 1, \parr, \bot)$ is a category $\category{C}$ consisting of:
		\begin{itemize}
			\item a monoidal category $(\category{C}, \otimes, 1, \alpha_\otimes, \lambda_\otimes, \rho_\otimes)$, with $\otimes$ called "tensor";
			\item a monoidal  category $(\category{C}, \parr, \bot, \alpha_\parr, \lambda_\parr, \rho_\parr)$, with $\parr$ called "par";
			\item two natural transformations called left and right \emph{linear distributors} respectively:
			\begin{align*}
				\partial_L:& A \otimes (B \parr C) \rightarrow (A\otimes B) \parr C
				\\
				\partial_R:& (A \parr B) \otimes C \rightarrow A \parr (B \otimes C)
			\end{align*}
		\end{itemize}
		satisfying the following coherence conditions:
		\begin{itemize}
			\item 
				coherence between the distributors and unitors:
				\begin{center}
					\begin{tikzcd}
						1 \otimes (A \parr B)
						\arrow[rr, "\partial_L " ]
						\arrow[rd, "\lambda_\otimes "' ]
						&
						&
						(1 \otimes A) \parr B
						\arrow[ld, "\lambda_\otimes \parr \id " ]
						\\
						&
						A \parr B
						&
					\end{tikzcd}
					\begin{tikzcd}
						(A \parr B) \otimes 1
						\arrow[rr, "\partial_R " ]
						\arrow[rd, " \rho_\otimes "' ]
						&
						&
						A \parr (B \otimes 1)
						\arrow[ld, "\id \otimes \rho_\otimes " ]
						\\
						&
						A \parr B
						&
					\end{tikzcd}
					
					\begin{tikzcd}
						A \otimes (B \parr \bot ) 
						\arrow[rr, "\partial_L " ]
						\arrow[dr, "\id \otimes \rho_\parr "' ]
						&
						&
						(A \otimes B) \parr \bot
						\arrow[dl, "\rho_\parr " ]
						\\
						&
						A\otimes B
						&
					\end{tikzcd}
					\begin{tikzcd}
						(\bot \parr A) \otimes B
						\arrow[rr, "\partial_R " ]
						\arrow[dr, "\lambda_\parr \otimes \id "' ]
						&
						&
						\bot \parr (A \otimes B)
						\arrow[dl, "\lambda_\parr " ]
						\\
						&
						A \otimes B
						&
					\end{tikzcd}
				\end{center}
	%			\formatnote{Diagramme besser alignen}
				%comment.............................................................................................................
			\item 
				coherence between distributors and associators:
				\begin{center}
					\begin{tikzcd}
						&
						(A \otimes B) \otimes (C \parr D)
						\arrow[rd, "\partial_L " ]
						\arrow[ld, "\alpha_\otimes " ]
						&
						\\
						A \otimes (B \otimes (C \parr D))
						\arrow[d, "\id \otimes \partial_L " ]
						
						&
						&
						((A \otimes B) \otimes C) \parr D
						\arrow[d, "\alpha_\otimes \parr \id " ]
						\\
						A \otimes ((B \otimes C) \parr D)
						\arrow[rr, "\partial_L " ]
						&
						&
						(A \otimes (B \otimes C)) \parr D
					\end{tikzcd}
					
					\begin{tikzcd}
						&
						A \otimes (B \parr (C \parr D))
						\arrow[ld, "\id \otimes \alpha_\parr " ]
						\arrow[rd, "\partial_L " ]
						&
						\\
						A \otimes ((B \parr C) \parr D)
						\arrow[d, "\partial_L " ]
						&
						&
						(A \otimes B) \parr (C \parr D)
						\arrow[d, "\alpha_\parr " ]
						\\
						(A \otimes (B \parr C)) \parr D
						\arrow[rr, "\partial_L \parr \id " ]
						&
						&
						((A \otimes B) \parr C) \parr D    \end{tikzcd}
					
					\begin{tikzcd}
						&
						(A \parr B) \otimes (C \otimes D)
						\arrow[ld, "\alpha_\otimes " ]
						\arrow[rd, "\partial_R " ]
						&
						\\
						((A \parr B) \otimes C) \otimes D
						\arrow[d, "\partial_R \otimes \id " ]
						&
						&
						A \parr (B \otimes (C \otimes D))
						\arrow[d, "\id \parr \alpha_\otimes " ]
						\\
						(A \parr (B \otimes C)) \otimes D
						\arrow[rr, "\partial_R " ]
						&
						&
						A \parr ((B \otimes C) \otimes D)
					\end{tikzcd}
					
					\begin{tikzcd}
						&
						((A \parr B) \parr C) \otimes D
						\arrow[ld, "\alpha_\parr \otimes \id " ]
						\arrow[rd, "\partial_R " ]
						&
						\\
						(A \parr (B \parr C)) \otimes D
						\arrow[d, "\partial_R " ]
						&
						&
						(A \parr B) \parr (C \otimes D)
						\arrow[d, "\alpha_\parr " ]
						\\
						A \parr ((B \parr C) \otimes D)
						\arrow[rr, "\id \parr \partial_R " ]
						&
						&
						A \parr (B \parr (C \otimes D))
					\end{tikzcd}
				\end{center}
			\item 
				coherence between the distributors:
				\begin{center}
					\begin{tikzcd}
						&
						(A \parr B) \otimes (C \parr D)
						\arrow[rd, "\partial_L " ]
						\arrow[ld, "\partial_R " ]
						&
						\\
						((A \parr B) \otimes C) \parr D
						\arrow[d, "\partial_R \parr \id " ]
						&
						&
						A \parr (B \otimes (C \parr D))
						\arrow[d, "\id \parr \partial_L " ]
						\\
						(A \parr (B \otimes C)) \parr D
						&
						&
						A \parr ((B \otimes C) \parr D)
						\arrow[ll, "\alpha_\parr " ]
					\end{tikzcd}
					%comment.............................................................................................................
					%\formatnote[inline]{obiges Diagramm rumdrehen}
					%comment.......................................................................................................
					%\formatnote[inline]{\textbackslash parr \textbackslash id sieht kacke aus...}    %..................................................................................
					\begin{tikzcd}
						&
						A \otimes ((B \parr C) \otimes D)
						\arrow[ld, "\alpha_\otimes " ]
						\arrow[rd, "\id \otimes \partial_R " ]
						&
						\\
						(A \otimes (B \parr C)) \otimes D
						\arrow[d, "\partial_L \otimes \id " ]
						&
						&
						A \otimes (B \parr (C \otimes D))
						\arrow[d, "\partial_L " ]
						\\
						((A \otimes B) \parr C) \otimes D
						\arrow[rr, "\partial_R " ]
						&
						&
						(A \otimes B) \parr (C \otimes D)
					\end{tikzcd}
				\end{center}
		\end{itemize}
	\end{definition}
	
	
	\subsection{*-autonomous Categories}
	\label{sec: autoCats}
	
	\begin{definition}[Dual object]
		\label{def: dualObj}
		%source: srinivasan
		Let $\category{C} $ be a linearly distributive category and $A, A^* \in \Obj(\category{C}) $, then we call $A^* $ \emph{left dual} (or \emph{left linearly adjoint}) to $A$, if there are morphisms $\tau: 1 \rightarrow A^* \parr A$ and $\gamma: A \otimes A^* \rightarrow \bot $, called \emph{unit} and \emph{counit} resp., such that the following diagrams commute:
		\begin{center}
			\begin{tikzcd}
				A^*
				\arrow[r, equal, "\id" ]
				&
				A^*
				\arrow[d, "\lambda_\otimes^{-1} " ]
				\\
				A^* \parr \bot
				\arrow[u, "\rho_\parr " ]
				&
				1 \otimes A^*
				\arrow[d, "\tau \otimes \id " ]
				\\
				A^*\parr (A \otimes A^*)
				\arrow[u, "\id \parr \gamma " ]
				&
				(A^* \parr A) \otimes A^*
				\arrow[l, "\partial_R " ]
			\end{tikzcd}
			\begin{tikzcd}
				A
				\arrow[r, equal, "\id " ]
				&
				A
				\arrow[d, "\rho_\otimes^{-1} " ]
				\\
				\bot \parr A
				\arrow[u, "\lambda_\parr " ]
				&
				A \otimes 1
				\arrow[d, "\id \otimes \tau " ]
				\\
				(A \otimes A^*) \parr A
				\arrow[u, "\gamma \parr \id " ]
				&
				A \otimes (A^* \parr A)
				\arrow[l, "\partial_L " ]
			\end{tikzcd}
		\end{center}
		
		We write $(\tau, \gamma): A^* \dashV A $ and also call $A$ \emph{right dual} to $A^*$. 
		
		%........................................................................................................................................................
		%\formatnote[inline]{mathabx incompatible with amssymb, stix absolutely broken; repalce dashV with dashv + some index?}
		%comment..................................................................................................................................................................................................................................................................
	\end{definition}
	
	\begin{lemma}
		\label{lemma: isoLinDisCat}
		
		\begin{enumerate}
			\item In an LDC: if 
			$A^* \dashV A $ and $A' \dashV A $, 
			then $A^*$ and $A'$ are isomorphic. 
			We will from now on only talk about \emph{the} dual object, when equality up to isomorphism is sufficient.
			\item In a symmetric LDC: 
			$(\tau, \gamma): A^* \dashV A 
			\Longleftrightarrow 
			(\tau s_{\parr}, s_{\otimes} \gamma): A \dashV A^* $ 
		\end{enumerate}
	\end{lemma}
	
%	\todo[inline]{proof }
	
%	\begin{proof}
		
%	\end{proof}
	
	%(BCS: "Feedback for linearly distributive categories: traces and fixpoints)
	%comment..............................................................................................................................
	
	%\todo[inline]{\cite{srinavsan-thesis} Lemma 2.9 (iii)?}
	%comment.............................................................................................................................
	
	\begin{definition}[*-autonomous categories \cite{srinavsan-thesis}]
		\label{def: autoCatSrinivasan}
		An LDC $\category{C} $ in which for every Object $A \in \category{C} $ there exists a left and right dual, resp. $(\tau*, \gamma*): A^* \dashV A $ and $(*\tau, *\gamma): A \dashV \leftindex^* {A} $, is called \emph{*-autonomous category}.
	\end{definition}
	
	
	%cut content......................................................................................................................................................
	\iffalse
	\begin{definition}[*-autonomous categories (\cite{barr95}, A)]
		\label{def: autoCatBarrA}
		A *-autonomous category is a biclosed monoidal category $\category{C} $ with a closed functor $(-)^*: \category{C} \rightarrow \category{C}^\opp $, which is a strong equivalence of categories.     
	\end{definition}
	%Barr, 95
	\todo[inline]{definition closed functor, strong equivalence. (strong closed functor?)}
	
	\begin{definition}[Dualizing object]
		\label{def: dualizingObj}
		Let $\category{C} $ be a biclosed monoidal category. 
		%\todo{details (Barr, 99)}
		An object $\bot $ is called a dualizing object if for every $A \in \Obj(\category{C}) $ the natural map $A \rightarrow \bot \multimapinv (A \multimap \bot) $ gotten by transposing $\id: A \multimap \bot \rightarrow A\multimap \bot $ twice is an isomorphism. 
	\end{definition}
	
	\begin{definition}[*-autonomous categories (Barr, B)]
		\label{def: autoCatBarrB}
		A *-autonomous category is a biclosed category with a dualizing object.
	\end{definition}
	
	
	\begin{definition}[*-autonomous categories (Barr, C)]
		\label{def: autoCatBarrC}
		A *-autonomous category is a monoidal category $\category{C} $ equipped with an equivalence $(-)^* : \category{C} \rightarrow \category{C}^\opp $ such that there is a natural isomorphism 
		\begin{align*}
			\Hom(A, B^*) \rightarrow \Hom(1, (A \oplus B)^*)
		\end{align*}
	\end{definition}
	\fi 



%__________________________________________________________________________________________________________________________________________________________________________________________________________________________________________________________________________________________________________________________
	
	\section{Linear Logic Preliminaries}
	
	
	%=============================================================================================================================
	\subsection{Syntax}
%	\urgentnote[inline]{Quellen, Quellen, Quellen!!!}
%	\todo[inline]{main source: IntroLinLog}
	%comments..............................................................................................
	
	
	\begin{definition}[Formula]
		Formulas are defined inductively from atomic formulas and four constants $1$, $\bot$, $\top$ and $0$:
		\begin{itemize}
			\item Every constant is a formula.
			\item Every atomic formula is a formula.
			\item If $A$ is a formula, so are $A^\bot$,  $\oc A$ and $\wn A$.
			\item If $A$ and $B$ are formulas, so are $A \otimes B$, $A \parr B$, $A \with B$ and $A \oplus B$.
		\end{itemize}
		%It is customary in linear logic to call $p^\bot$ atomic, for an atom $p$.
	\end{definition}
	
	Let $\Gamma $, $\Delta$ etc. be arbitrary, finite lists of formulas (e.g.: $\Gamma = p_1, ..., p_n$), and $A$, $B$ etc. formulas.
	
	As we will later consider fragments without negation, we shall define linear logic with a two-sided calculus. 
	
	
	\begin{definition}[Proof (tree)]
		A \emph{proof tree} (often just \emph{proof}) is a rooted tree with sequents of the form $\Gamma \vdash \Delta $ as \emph{edges} and the following sequent rules as \emph{vertices} (except for the root).
	\end{definition}
	
	\begin{remark}
		Normally, proof trees are not written in a very "graph-like" way, which might obfuscate their nature as trees. The proof tree
		\begin{center}
			\AxiomC{\strut}
			\RightLabel{$\id$}
			\UnaryInfC{$A \vdash A $}
			\AxiomC{\strut}
			\RightLabel{$\id$}
			\UnaryInfC{$B \vdash B $}
			\RightLabel{$\otimes_R $}
			\BinaryInfC{$A, B \vdash A \otimes B$}
			\RightLabel{$\otimes_L $}
			\UnaryInfC{$A \otimes B \vdash A \otimes B $}
			\DisplayProof
		\end{center}
		can be displayed more graph-like as
		\begin{center}
			\begin{tikzpicture}
				\begin{scope}[every node/.style={circle,draw}]
					\node (id1) at (0,0) {id};
					\node (id2) at (4,0) {id};
					\node (otimesR) at (2,-2) { $\otimes_R$ };
					\node (otimesL) at (2,-4) {$\otimes_L $};
					\node (root) at (2,-6) {$\surd$};
				\end{scope}
				\begin{scope}
					\path [-] (id1) edge node[left] {$A \vdash A $} (otimesR);
					\path [-] (id2) edge node[right] {$B \vdash B $} (otimesR);
					\path [-] (otimesR) edge node[right] {$A, B \vdash A \otimes B$} (otimesL);
					\path [-] (otimesL) edge node[right] {$A \otimes B \vdash A \otimes B$} (root);
				\end{scope}
			\end{tikzpicture}
		\end{center}
		with $\surd$ marking the root. As this is neither efficient nor pretty, we will stick with the traditional display.
	\end{remark}
	
	
	\paragraph{Structural rules}
	We only have the exchange rule as a structural rule, missing the (general) weakening and contraction rules known from classical logic:
	
	\begin{center}
		\AxiomC{$\Gamma, A, B, \Gamma' \vdash \Delta$}
		\RightLabel{ex.L}
		\UnaryInfC{$\Gamma, B, A, \Gamma' \vdash \Delta$}
		\DisplayProof
		\quad
		\AxiomC{$\Gamma \vdash \Delta , A, B, \Delta'$}
		\RightLabel{ex.R}
		\UnaryInfC{$\Gamma \vdash \Delta , B, A, \Delta' $}
		\DisplayProof
	\end{center}
	
	
	\paragraph{Identity rules}
	We have the identity and negation rules:
	
	\begin{center}
		\AxiomC{\strut} %temporäre Lösung (auch an anderen Stellen verwendet!)
		\RightLabel{id}
		\UnaryInfC{$A \vdash A$}
		\DisplayProof
		\quad
		\AxiomC{$\Gamma_1 \vdash \Delta_1 , A ,  \Delta'_1$}
		\AxiomC{$\Gamma_2 , A , \Gamma'_2 \vdash \Delta_2 $}
		\RightLabel{Cut}
		\BinaryInfC{$\Gamma_2 , \Gamma_1 , \Gamma'_2 \vdash \Delta_1 , \Delta_2 , \Delta'_1 $}
		\DisplayProof
		
		\AxiomC{$\Gamma \vdash A, \Delta $}
		\RightLabel{neg.L}
		\UnaryInfC{$\Gamma, A^\bot \vdash \Delta $}
		\DisplayProof
		\quad
		\AxiomC{$\Gamma, A \vdash \Delta $}
		\RightLabel{neg.R}
		\UnaryInfC{$\Gamma \vdash A^\bot, \Delta $}
		\DisplayProof
	\end{center}
	
	
	
	As already mentioned, the classical conjunction $\wedge$ and disjunction $\vee$ as well as their respective units split into two respectively. These can be classified as multiplicative and additive connectives. 
	
	\paragraph{Multiplicatives }
	The calculus rules for the \emph{multiplicative} conjunction $\otimes$, disjunction $\parr$ ("par") and their units $1$ and $\bot$ ("bottom") are as follows:
	
	\begin{center}
		\AxiomC{$\Gamma , A, B, \Gamma' \vdash \Delta$}
		\RightLabel{$\otimes_L$}
		\UnaryInfC{$\Gamma, A\otimes B, \Gamma' \vdash \Delta $}
		\DisplayProof
		\quad
		\AxiomC{$\Gamma \vdash \Delta , A$}
		\AxiomC{$\Gamma' \vdash B, \Delta'$}
		\RightLabel{$\otimes_R$}
		\BinaryInfC{$\Gamma, \Gamma' \vdash \Delta , A\otimes B, \Delta'$}
		\DisplayProof
		
		
		\AxiomC{$\Gamma , A \vdash \Delta $}
		\AxiomC{$B , \Gamma' \vdash \Delta' $}
		\RightLabel{$\parr_L$}
		\BinaryInfC{$\Gamma, A\parr B, \Gamma' \vdash \Delta, \Delta'$}
		\DisplayProof
		\quad
		\AxiomC{$\Gamma \vdash \Delta , A, B,  \Delta'$}
		\RightLabel{$\parr_R$}
		\UnaryInfC{$\Gamma \vdash \Delta , A \parr B, \Delta' $}
		\DisplayProof
		
		\AxiomC{$\Gamma \vdash \Delta $}
		\RightLabel{$1_L$}
		\UnaryInfC{$\Gamma_1, 1, \Gamma_2 \vdash \Delta $}
		\DisplayProof
		\quad
		\AxiomC{\vphantom{$\frac{a}{b}$ } }
		\RightLabel{$1_R $}
		\UnaryInfC{$\vdash 1$}
		\DisplayProof
		
		\AxiomC{\vphantom{$\frac{a}{b}$}}
		\RightLabel{$\bot_L$}
		\UnaryInfC{$\bot \vdash $}
		\DisplayProof
		\quad
		\AxiomC{$\Gamma \vdash \Delta $ }
		\RightLabel{$\bot_R $}
		\UnaryInfC{$\Gamma \vdash \Delta_1, \bot , \Delta_2 $}
		\DisplayProof
	\end{center}
	Note that $\Gamma = \Gamma_1 \Vert \Gamma_2 $ and $\Delta = \Delta_1 \Vert \Delta_2$ for $1_L$ and $\bot_R$.
	
	
	\begin{remark}
		The rules $\otimes_L$ and $\parr_R$ imply, that the commas are to be read as $\otimes$ on the left-hand side and as $\parr$ on the the right-hand side. That means $A, B \vdash C, D$ is provable iff $A \otimes B \vdash C \parr D$ is provable. 
	\end{remark}
	
	\begin{proof}
		We only have to show that $A , B \vdash C , D $ follows from $A\otimes B \vdash C \parr D$ as the other direction is just our introduction rule:
		
		\begin{center}
			\AxiomC{}
			\RightLabel{id}
			\UnaryInfC{$A \vdash A $}
			\AxiomC{}
			\RightLabel{id}
			\UnaryInfC{$B \vdash B $}
			\RightLabel{$\otimes_R$}
			\BinaryInfC{$A , B \vdash A \otimes B $}
			
			\AxiomC{$A \otimes B \vdash C \parr D $}
			
			\RightLabel{Cut}
			\BinaryInfC{$A , B \vdash C \parr D $}
			
			\AxiomC{}
			\UnaryInfC{$C \vdash C $}
			\AxiomC{}
			\UnaryInfC{$D \vdash D $}
			\RightLabel{$\parr_L $}
			\BinaryInfC{$C \parr D \vdash C , D $}
			\BinaryInfC{$A  , B \vdash C , D$}
			\DisplayProof
		\end{center}
		
		\phantom{blabla}
	\end{proof}
	
	\paragraph{Additives}
	The calculus rules for the \emph{additive} conjunction $\with$ and disjunction $\oplus$ ("plus") are as follows:
	
	\begin{center}
		\AxiomC{$\Gamma , A \vdash \Delta $}
		\RightLabel{$\with_{L1}$}
		\UnaryInfC{$\Gamma , A \with B \vdash \Delta $}
		\DisplayProof
		\quad
		\AxiomC{$\Gamma , B \vdash \Delta $}
		\RightLabel{$\with_{L2}$}
		\UnaryInfC{$\Gamma , A \with B \vdash \Delta $}
		\DisplayProof
		
		\AxiomC{$\Gamma \vdash \Delta , A $}
		\AxiomC{$\Gamma \vdash \Delta , B $}
		\RightLabel{$\with_R$}
		\BinaryInfC{$\Gamma \vdash \Delta , A \with B $}
		\DisplayProof
	\end{center}
	
	\begin{center}
		\AxiomC{$\Gamma , A \vdash \Delta $}
		\AxiomC{$ \Gamma , B \vdash \Delta $}
		\RightLabel{$\oplus_L$}
		\BinaryInfC{$\Gamma , A\oplus B \vdash \Delta$}
		\DisplayProof
		
		\AxiomC{$\Gamma \vdash A , \Delta $}
		\RightLabel{$\oplus_{R1}$}
		\UnaryInfC{$\Gamma \vdash A \oplus B , \Delta $}
		\DisplayProof
		\quad
		\AxiomC{$\Gamma \vdash B , \Delta $}
		\RightLabel{$\oplus_{R2}$}
		\UnaryInfC{$\Gamma \vdash A \oplus B ,  \Delta $}
		\DisplayProof
		
		\AxiomC{\vphantom{$\frac{a}{b}$}}
		\RightLabel{$0_L$ }
		\UnaryInfC{$\Gamma, 0 \vdash \Delta $}
		\DisplayProof
		\quad
		\AxiomC{\vphantom{$\frac{a}{b}$}}
		\RightLabel{$\top_R$ }
		\UnaryInfC{$\Gamma \vdash \top , \Delta $}
		\DisplayProof
	\end{center}
	
	Notice the difference between $\with_R$ and $\otimes_R$ (and dually between $\oplus_L$ and $\parr_L$): while for $\otimes_R$ the contexts $\Gamma$ etc. are arbitrary and get combined in the conclusion, $\with_R$ requires the contexts to be equal. In classical logic these rules can be shown to be equivalent using its additional structural rules.
	
	Furthermore, note that the unit introduction rules also introduce arbitrary contexts, and that each unit only has one introduction rule as opposed to the multiplicative units.
	
	
	
	\begin{remark}
		Similarly to the multiplicative connectives above, the additive connectives have a invertibility statement: $\Gamma \vdash A \with B $ is provable iff $\Gamma \vdash A$ and $\Gamma \vdash B$ are provable. Dually, $A \oplus B \vdash \Delta $ iff $A \vdash \Delta $ and $B \vdash \Delta $.
		
	\end{remark}
	
	\begin{proof}
		As with the multiplicative statement, one direction suffices:
		\begin{center}
			\AxiomC{$\Gamma \vdash A \with B$}
			
			\AxiomC{}
			\UnaryInfC{$A \vdash A$}
			\UnaryInfC{$A \with B \vdash A$}
			\BinaryInfC{$\Gamma \vdash A$}
			\DisplayProof
		\end{center}
		
		$\Gamma \vdash B$ follows the same way.
	\end{proof}
	
	
	
	\begin{definition}
		We call two formulas $A$ and $B $ \emph{(linearly) equivalent} iff $A \vdash B$ and $B \vdash A $ are provable, and write $A \equiv B$.
	\end{definition}
	
	\begin{remark}
		The names "multiplicative" and "additive" are motivated by the following relations:
		
		\begin{align*}
			A \otimes (B \oplus C) 
			& \equiv (A \otimes B) \oplus (A \otimes C)
			\\
			A \parr (B \with C ) 
			& \equiv (A \parr B) \with (A \parr C )
		\end{align*}
		
		The similarities to basic arithmetic don't end there:
		\begin{align*}
			A \otimes 0 \equiv 0 ,
			\quad 
			A \parr \top \equiv \top
		\end{align*}
	\end{remark}
	
	
	\begin{proof}
		We only show $A \otimes (B \oplus C) \equiv (A \otimes B) \oplus (A \otimes C)$% and $A \otimes 0 \equiv 0$
		.
		\begin{itemize}
			\item $A \otimes (B \oplus C) \vdash (A \otimes B) \oplus (A \otimes C)$:
			\begin{center}
				\AxiomC{}
				\UnaryInfC{$A \vdash A$}
				\AxiomC{}
				\UnaryInfC{$B \vdash B $}
				\BinaryInfC{$A, B 
					\vdash A \otimes B $}
				\UnaryInfC{$A , B 
					\vdash (A \otimes B) \oplus (A \otimes C ) $}
				\AxiomC{}
				\UnaryInfC{$A \vdash A$}
				\AxiomC{}
				\UnaryInfC{$C \vdash C$}
				\BinaryInfC{$A, C 
					\vdash A \otimes C $}
				\UnaryInfC{$A , C 
					\vdash (A \otimes B) \oplus (A \otimes C ) $}
				\BinaryInfC{$A, B \oplus C 
					\vdash (A \otimes B) \oplus (A \otimes C ) $}
				\UnaryInfC{$A \otimes (B \oplus C) 
					\vdash (A \otimes B) \oplus (A \otimes C ) $}
				\DisplayProof
			\end{center}
			
			\item $(A \otimes B) \oplus (A \otimes C) \vdash A \otimes (B \oplus C) $:
			\begin{center}
				\AxiomC{}
				\UnaryInfC{$A \vdash A$}
				\AxiomC{}
				\UnaryInfC{$B \vdash B $}
				\UnaryInfC{$B \vdash B \oplus C$}
				\BinaryInfC{$A, B 
					\vdash A \otimes (B \oplus C ) $}
				\UnaryInfC{$A \otimes B \vdash A \otimes (B \oplus C ) $}
				\AxiomC{}
				\UnaryInfC{$A \vdash A$}
				\AxiomC{}
				\UnaryInfC{$C \vdash C $}
				\UnaryInfC{$C \vdash B \oplus C$}
				\BinaryInfC{$A, C 
					\vdash A \otimes (B \oplus C ) $}
				\UnaryInfC{$A \otimes C \vdash A \otimes (B \oplus C ) $}
				\BinaryInfC{$(A \otimes B) \oplus (A \otimes C) \vdash A \otimes (B \oplus C) $}
				\DisplayProof
			\end{center}
			
			%			\item $A \otimes 0 \vdash 0$:
			%			\begin{center}
				%				\AxiomC{}
				%				\RightLabel{$0_L$}
				%				\UnaryInfC{$A, 0 \vdash 0$}
				%				\UnaryInfC{$A \otimes 0 \vdash 0$}
				%				\DisplayProof
				%			\end{center}
			%			
			%			\item $0 \vdash A \otimes 0$
			%			\begin{center}
				%				\AxiomC{}
				%				\UnaryInfC{$0 \vdash 0 \otimes A$}
				%				\DisplayProof
				%			\end{center}
		\end{itemize}
		The proof for $A \parr (B \with C ) \equiv (A \parr B) \with (A \parr C )$ is similar, the proof for the equations of the constants a trivial application of their introduction rules.
	\end{proof}
	
	
	\begin{theorem}
		We have the following equivalencies for the linear negation:
		\begin{itemize}
			\item For the constants: 
			\begin{align*}
				1^\bot \equiv \bot \quad 
				& 
				\bot^\bot \equiv 1 
				\\
				\top^\bot \equiv \quad 
				& 
				0^\bot \equiv \top
			\end{align*}
			\item The negation is involutory: 
			$(A^\bot)^\bot \equiv A $
			\item The De Morgan equations hold:
			\begin{align*}
				\left(A \otimes B\right)^\bot \equiv A^\bot \parr B^\bot 
				\quad
				&
				\left(A \parr B\right)^\bot \equiv A^\bot \otimes B^\bot 
				\\
				\left(A \with B\right)^\bot \equiv A^\bot \oplus B^\bot 
				\quad
				&
				\left(A \oplus B\right)^\bot \equiv A^\bot \with B^\bot 
			\end{align*}
		\end{itemize}
	\end{theorem}
	
	\begin{proof}
		We shall only prove parts.
		
		\begin{itemize}
			\item $1^\bot \vdash \bot $: 
			\begin{center}
				\AxiomC{\strut}
				\RightLabel{$1_R$}
				\UnaryInfC{$\vdash 1 $}
				\RightLabel{$\bot_R$}
				\UnaryInfC{$\vdash 1, \bot $}
				\RightLabel{neg.L}
				\UnaryInfC{$1^\bot \vdash \bot $}
				\DisplayProof
			\end{center}
			\item $(A^\bot)^\bot \vdash A $:
			\begin{center}
				\AxiomC{\strut}
				\UnaryInfC{$A \vdash A $}
				\UnaryInfC{$\vdash A^\bot, A $}
				\UnaryInfC{$(A^\bot)^\bot \vdash A $}
				\DisplayProof
			\end{center}
			\item $\left(A \otimes B\right)^\bot \vdash A^\bot \parr B^\bot $:
			\begin{center}
				\AxiomC{\strut}
				\UnaryInfC{$A \vdash A $}
				\UnaryInfC{$\vdash A^\bot, A $}
				\AxiomC{\strut}
				\UnaryInfC{$B \vdash B $}
				\UnaryInfC{$\vdash B^\bot, B $}
				\BinaryInfC{$\vdash A \otimes B, A^\bot, B^\bot $}
				\UnaryInfC{$\left(A \otimes B\right)^\bot \vdash A^\bot, B^\bot $}
				\UnaryInfC{$\left(A \otimes B\right)^\bot \vdash A^\bot \parr B^\bot $}
				\DisplayProof
			\end{center}
		\end{itemize}
		The rest is shown similarly.
	\end{proof}
	
	With the negation our calculus rules become quite redundant and we can restrict them on the rules for the right-hand side by translating any two-sided sequent 
	$A_1, ..., A_n \vdash B_1, ..., B_n$ 
	into a one-sided sequent 
	$\vdash A_1^\bot, ... A_n^\bot, B_1, ... B_n$. 
	However, these redundancies are necessary when dealing with a negation-free fragment of LL as we will later do.
	
	
	\begin{definition}[Syntactical implication and equivalence]
		We define linear implication with the par-operator:
		\begin{align*}
			A \multimap B := A^\bot \parr B
		\end{align*}
		
		We further define (syntactical) linear equivalence:
		\begin{align*}
			A \multimapboth B := (A \multimap B) \with (B \multimap A)
		\end{align*}
		It is easily seen that $\vdash A\multimap B$ iff $A \vdash B$ and that $\vdash A \multimapboth B$ iff $A \equiv B$. 
	\end{definition}
	
	\paragraph{Exponentials }
	The modality connectives reintroduce stable truths and with them the the weakening and contraction rules known from classical logic:
	
	\begin{center}
		\AxiomC{$\Gamma \vdash \Delta$}
		\RightLabel{$\oc_W$ (weakening)}
		\UnaryInfC{$\Gamma , \oc A \vdash \Delta $}
		\DisplayProof
		\quad
		\AxiomC{$\Gamma, A \vdash \Delta $}
		\RightLabel{$\oc_D$ (dereliction)}
		\UnaryInfC{$\Gamma, \oc A \vdash \Delta $}
		\DisplayProof
		
		\AxiomC{$\Gamma , \oc A, \oc A \vdash \Delta $}
		\RightLabel{$\oc_C$ (contraction)}
		\UnaryInfC{$\Gamma , \oc A \vdash \Delta $}
		\DisplayProof
		\quad
		\AxiomC{$\oc\Gamma \vdash \wn\Delta , A$}
		\RightLabel{$\oc_R$}
		\UnaryInfC{$\oc\Gamma \vdash \wn\Delta , \oc A$}
		\DisplayProof
		
		
		\AxiomC{$\Gamma \vdash \Delta $}
		\RightLabel{$\wn_W$ (weakening)}
		\UnaryInfC{$\Gamma \vdash \Delta , \wn A$}
		\DisplayProof
		\quad
		\AxiomC{$\Gamma \vdash \Delta , A$}
		\RightLabel{$\wn_D$ (dereliction)}
		\UnaryInfC{$\Gamma \vdash \Delta , \wn A$}\
		\DisplayProof
		
		\AxiomC{$\Gamma \vdash \Delta , \wn A , \wn A $}
		\RightLabel{$\wn_C$ (contraction)}
		\UnaryInfC{$\Gamma \vdash \Delta , \wn A$}
		\DisplayProof
		\quad
		\AxiomC{$\oc \Gamma, A \vdash \wn \Delta $}
		\RightLabel{$\wn_L$}
		\UnaryInfC{$\oc \Gamma , \wn A \vdash \wn \Delta $}
		\DisplayProof
		
		\begin{align*}
			\left(\oc A \right) ^\bot \equiv \wn \left( A^\bot \right) 
			\quad 
			\left(\wn A \right) ^\bot \equiv \oc \left( A^\bot \right) 
		\end{align*}
	\end{center}
	Here, the context $\oc \Gamma $ is given by applying the $\oc $-modality to every formula of $\Gamma $, i.e. $\oc \Gamma = \oc q, \oc p, \ldots $ for $\Gamma = q, p \ldots $; same for $\wn \Delta$.
	
	
	\begin{remark}
		These modalities are called exponentials because of the	following relation:
		\begin{align*}
			\oc\left(A \with B\right) \equiv \oc A \otimes \oc B ,
			\quad 
			\wn \left(A \oplus B \right) \equiv \wn A \parr \wn B
		\end{align*}
	\end{remark}
	
	\begin{proof}
		We will only prove the second equivalence.
		\begin{itemize}
			\item 
			$\wn A \parr \wn B \vdash \wn (A \oplus B )$:
			\begin{center}
				\AxiomC{}
				\UnaryInfC{$A \vdash A$}
				\RightLabel{$\oplus_R $}
				\UnaryInfC{$A \vdash A\oplus B$}
				\RightLabel{$\wn_D $}
				\UnaryInfC{$A \vdash \wn (A\oplus B) $}
				\RightLabel{$\wn_L $}
				\UnaryInfC{$\wn A \vdash \wn (A\oplus B) $}
				
				\AxiomC{}
				\UnaryInfC{$B \vdash B$}
				\RightLabel{$\oplus_R $}
				\UnaryInfC{$B \vdash A\oplus B$}
				\RightLabel{$\wn_D $}
				\UnaryInfC{$B \vdash, \wn (A\oplus B) $}
				\RightLabel{$\wn_L $}
				\UnaryInfC{$\wn B \vdash \wn (A\oplus B) $}
				
				\RightLabel{$\parr $}
				\BinaryInfC{$(\wn A) \parr (\wn B) \vdash \wn (A\oplus B) , \wn (A\oplus B) $}
				\RightLabel{$\wn_D $}
				\UnaryInfC{$\wn A \parr \wn B \vdash \wn (A \oplus B)$}
				\DisplayProof
			\end{center}
			
			\item 
			$\wn (A \oplus B) \vdash \wn A \parr \wn B $:
			\begin{center}
				\AxiomC{}
				\UnaryInfC{$A \vdash A$}
				\RightLabel{$\wn_D$}
				\UnaryInfC{$A \vdash \wn A$}
				\RightLabel{$\wn_W$}
				\UnaryInfC{$A \vdash \wn A, \wn B $}
				
				\AxiomC{}
				\UnaryInfC{$B \vdash B$}
				\UnaryInfC{$B \vdash \wn B$}
				\RightLabel{$\wn_W$}
				\UnaryInfC{$B \vdash \wn B , \wn A $}
				
				\RightLabel{$\oplus_L $}
				\BinaryInfC{$ A \oplus B \vdash \wn A , \wn B $}
				\RightLabel{$\wn_L $}
				\UnaryInfC{$\wn (A \oplus B) \vdash \wn A, \wn B $}
				\RightLabel{$\parr_R$}
				\UnaryInfC{$\wn (A \oplus B) \vdash \wn A \parr \wn B $}
				\DisplayProof
			\end{center}
		\end{itemize}
		$\oc\left(A \with B\right) \equiv \oc A \otimes \oc B $ is acquired the same way.
	\end{proof}
	
%	\formatnote{proofsymbol なんで? :( }
	
	
	
	%================================================================================================================================================
	\subsection{Cut elimination }
	\label{cut-elimination}
	
	As is the case with most, if not all, logical system, the Cut rule of our calculus is admissible. 
	That means for any proof $\pi$ of a sequent $\Gamma \vdash \Delta $ we can construct a \emph{cut-free} proof $\pi'$ of that same sequent.  
	We will prove this for the multiplicative fragment by simplifying the proof due to Braüner \cite[Appendix B]{brauner}. 
	The transformations used in this construction will later form the basis for the categorical semantics of this fragment. 
	Braüner's proof includes the exponentials which require additional care because of the contraction rule. 
	The general strategy consists of either, moving a cut upwards in a proof or replacing it with cuts of simpler formulas. 
	To do this, we need two definitions:
	
	%	As is the case with most, if not all, logical system, the Cut rule of our calculus is admissible. That means for any proof $\pi$ of a sequent $\Gamma \vdash \Delta $ we can find a \emph{cut-free} proof $\pi'$ of that same sequent. We will prove this for the multiplicative fragment by adapting the cut-elimination procedure for intuitionistic linear logic by 
	%	\urgentnote{add literature} 
	%	Melliès to (multiplicative) classical linear logic and expanding a little bit on the proof given by Braüner \cite{brauner}.
	
	
	\begin{definition}[Degree]
		The \emph{degree} $\partial(A)$ of a formula $A$ is defined inductively:
		\begin{itemize}
			\item $\partial(A) = 1$, for $A$ atomic or constant
			\item $\partial \left(A^\bot\right) = \partial(A)$ 
			\item $\partial(A \otimes B) = \partial(A \parr B) = \max\{\partial(A), \partial(B)\} + 1$
			\item $\partial(\oc A ) = \partial(\wn A) = \partial(A) + 1$
		\end{itemize} 
		The degree of a cut is the degree of the cut formula. The cut formula is called \emph{principal formula}. The degree $\partial(\pi)$ of a proof $\pi$ is the supremum of the degrees of the cuts in the proof. 
	\end{definition}
	
	\begin{definition}[Height]
		The \emph{height} $h(\pi)$ of a proof $\pi$ is just its height as a rooted tree: 
		\begin{itemize}
			\item $h(\pi) = 1$ when $\pi$ is an instance of a rule with zero premises, i.e. an axiom.
			\item $h(\pi) = h(\tau) +1$, for $\pi = \AxiomC{$\tau$} \UnaryInfC{$\Gamma \vdash \Delta$} \DisplayProof$
			\item $h(\pi) = \max\{h(\tau), h(\tau')\} + 1$, for $\pi = \AxiomC{$\tau$} \AxiomC{$\tau'$} \BinaryInfC{$\Gamma \vdash \Delta$} \DisplayProof$
		\end{itemize}
	\end{definition}
	
	With these, we can now eliminate cuts in a proof:
	
	\begin{lemma}
		\label{key-cases}
		Let $\tau$ be a proof consisting of two subproofs $\tau_1$ and $\tau_2$ of strictly lower degree, i.e.: 
		\begin{center}
			$\tau = $ 
			\AxiomC{$\tau_1$} 
			\noLine
			\UnaryInfC{$\svdots$}
			\UnaryInfC{$\Gamma_1 \vdash \Delta_1, A, \Delta'_1 $} 
			\AxiomC{$\tau_2$} 
			\noLine
			\UnaryInfC{$\svdots$}
			\UnaryInfC{$\Gamma_2, A, \Gamma'_2 \vdash \Delta_2 $} 
			\BinaryInfC{$\Gamma_2, \Gamma_1, \Gamma'_2 \vdash \Delta_1, \Delta_2, \Delta'_1 $}
			\DisplayProof
		\end{center}
		
		Then, we can find a proof $\tau'$ of that same sequent 
		$\Gamma_2, \Gamma_1, \Gamma'_2 \vdash \Delta_1, \Delta_2, \Delta'_1 $ 
		with 
		$\partial(\tau') < \partial(\tau) $.
	\end{lemma}
	
	\begin{remark}
		We write 
		\AxiomC{$\pi$}
		\RightLabel{$r$}
		\UnaryInfC{$\Gamma \vdash \Delta$}
		\DisplayProof
		and 
		\AxiomC{$\pi$}
		\AxiomC{$\pi'$}
		\RightLabel{$r$}
		\BinaryInfC{$\Gamma \vdash \Delta$}
		\DisplayProof
		for the proof that we get when applying the rule $r$ on the last sequent(s) of the proof(s) $\pi$ (and $\pi'$).
		Furthermore, for the proof $\tau$ that ends in the sequent $\Gamma \vdash \Delta$, we write
		\AxiomC{$\tau$} 
		\noLine
		\UnaryInfC{$\svdots$}
		\UnaryInfC{$\Gamma \vdash \Delta$}
		\DisplayProof
		. 
	\end{remark}
	
	
	\begin{proof}
		Induction on $h(\tau_1) + h(\tau_2)$. The immediate subproofs of $\tau_i$ are denoted by $\pi_j$.
		We perform transformations case by case:
		
		
		\begin{itemize}
			\item If one of the the subproofs is an axiom, we just take the remaining one, resulting in a proof with lower degree:
			
			\begin{tabularx}{\textwidth}{x c x}
				\AxiomC{}
				\UnaryInfC{$A \vdash A $}
				
				\AxiomC{$\tau_2$}
				\noLine
				\UnaryInfC{$\svdots$}
				\UnaryInfC{$A, \Gamma \vdash \Delta $}
				
				\BinaryInfC{$A, \Gamma \vdash \Delta $}
				\DisplayProof
				&
				$\overset{(\id , -)}{\rightsquigarrow}$
				&
				\AxiomC{$\tau_2$}
				\noLine
				\UnaryInfC{$\svdots$}
				\UnaryInfC{$A, \Gamma \vdash \Delta $}
				\DisplayProof
				
				\\
				%......................................
				
				\AxiomC{$\tau_1$}
				\noLine
				\UnaryInfC{$\svdots$}
				\UnaryInfC{$\Gamma \vdash \Delta, A $}
				
				\AxiomC{}
				\UnaryInfC{$A \vdash A $}
				
				\BinaryInfC{$\Gamma \vdash \Delta, A $}
				\DisplayProof
				&
				$\overset{(-, \id)}{\rightsquigarrow}$
				&
				\AxiomC{$\tau_1$}
				\noLine
				\UnaryInfC{$\svdots$}
				\UnaryInfC{$\Gamma \vdash \Delta, A $}
				\DisplayProof
			\end{tabularx}
			
			\item $\tau_1$ does not introduce the principal formula in its last rule. Depending on the subcase, we perform one of the following transformations, pushing $\tau_2$ upwards, and use the induction hypothesis on the resulting subproof.
			
			
			
			\begin{tabularx}{\textwidth}{x c x}
				\scalebox{0.6}{
					\AxiomC{$\pi_1$}
					\UnaryInfC{$\Gamma_1 \vdash \Delta_1, C, \Delta'_1, A $}
					
					\AxiomC{$\pi_2$}
					\UnaryInfC{$\Gamma_2 \vdash B, \Delta_2 $}
					
					\RightLabel{$\otimes_R$}
					\BinaryInfC{$\Gamma_1,\Gamma_2 \vdash \Delta_1, C, \Delta'_1, A \otimes B, \Delta_2 $}
					
					\AxiomC{$\tau_2$}
					\noLine
					\UnaryInfC{$\svdots$}
					\UnaryInfC{$\Gamma_3, C, \Gamma'_3 \vdash \Delta_3$}
					
					\RightLabel{Cut}
					\BinaryInfC{$\Gamma_3, \Gamma_1, \Gamma_2, \Gamma'_3 \vdash \Delta_1, \Delta_3, \Delta'_1, A \otimes B, \Delta_2$}
					\DisplayProof
				}
				&
				$\overset{(\downarrow \otimes_R, -)}{\rightsquigarrow}$
				&
				\scalebox{0.6}{
					\AxiomC{$\pi_1$}
					\UnaryInfC{$\Gamma_1 \vdash \Delta_1, C, \Delta'_1, A $}
					
					\AxiomC{$\tau_2$}
					\noLine
					\UnaryInfC{$\svdots$}
					\UnaryInfC{$\Gamma_3, C, \Gamma'_3 \vdash \Delta_3 $}
					
					\RightLabel{Cut}
					\BinaryInfC{$\Gamma_3,\Gamma_1, \Gamma'_3 \vdash \Delta_1, \Delta_3, \Delta'_1, A $}
					
					\AxiomC{$\pi_2$}
					\UnaryInfC{$\Gamma_2 \vdash B, \Delta_2 $}
					
					\RightLabel{$\otimes_R$}
					\BinaryInfC{$\Gamma_3, \Gamma_1, \Gamma'_3, \Gamma_2 \vdash \Delta_1, \Delta_3, \Delta'_1, A \otimes B, \Delta_2 $}
					\doubleLine
					\RightLabel{ex.}
					\UnaryInfC{$\Gamma_3, \Gamma_1, \Gamma_2, \Gamma'_3 \vdash \Delta_1, \Delta_3, \Delta'_1, A \otimes B, \Delta_2 $}
					\DisplayProof
				}
				
				\\
				%...................................
				
				\scalebox{0.6}{
					\AxiomC{$\pi_1$}
					\UnaryInfC{$\Gamma_1 \vdash \Delta_1, A $}
					
					\AxiomC{$\pi_2$}
					\UnaryInfC{$\Gamma_2 \vdash B, \Delta_2, C, \Delta'_2, $}
					
					\RightLabel{$\otimes_R$}
					\BinaryInfC{$\Gamma_1,\Gamma_2 \vdash \Delta_1, A \otimes B, \Delta_2, C, \Delta_2' $}
					
					\AxiomC{$\tau_2$}
					\noLine
					\UnaryInfC{$\svdots$}
					\UnaryInfC{$\Gamma_3, C, \Gamma'_3 \vdash \Delta_3$}
					
					\RightLabel{Cut}
					\BinaryInfC{$\Gamma_3, \Gamma_1, \Gamma_2, \Gamma'_3 \vdash \Delta_1, A \otimes B, \Delta_2, \Delta_3, \Delta_2' $}
					\DisplayProof
				}
				&
				$\overset{(\downarrow \otimes_R, -)}{\rightsquigarrow}$
				&
				\scalebox{0.6}{
					\AxiomC{$\pi_1$}
					\UnaryInfC{$\Gamma_1 \vdash \Delta_1, A $}
					
					\AxiomC{$\pi_2$}
					\UnaryInfC{$\Gamma_2 \vdash B, \Delta_2, C, \Delta_2' $}
					
					\AxiomC{$\tau_2$}
					\noLine
					\UnaryInfC{$\svdots$}
					\UnaryInfC{$\Gamma_3, C, \Gamma'_3 \vdash \Delta_3 $}
					
					\RightLabel{Cut}
					\BinaryInfC{$\Gamma_3,\Gamma_2, \Gamma'_3 \vdash B, \Delta_2, \Delta_3, \Delta_2' $}
					
					\RightLabel{$\otimes_R$}
					\BinaryInfC{$\Gamma_1, \Gamma_3, \Gamma_2, \Gamma'_3 \vdash \Delta_1, A\otimes B, \Delta_2, \Delta_3, \Delta_2' $}
					\doubleLine
					\RightLabel{ex.}
					\UnaryInfC{$\Gamma_3, \Gamma_1, \Gamma_2, \Gamma'_3 \vdash \Delta_1, A \otimes B, \Delta_2, \Delta_3, \Delta_2' $}
					\DisplayProof
				}
				
				\\
				%...................................
				
				\scalebox{0.6}{
					\AxiomC{$\pi_1$}
					\UnaryInfC{$\Gamma_1, A \vdash \Delta_1, C, \Delta'_1 $}
					
					\AxiomC{$\pi_2$}
					\UnaryInfC{$B, \Gamma_2 \vdash \Delta_2 $}
					
					\RightLabel{$\parr_L$}
					\BinaryInfC{$\Gamma_1, A \parr B, \Gamma_2 \vdash \Delta_1, C, \Delta'_1, \Delta_2 $}
					
					\AxiomC{$\tau_2$}
					\noLine
					\UnaryInfC{$\svdots$}
					\UnaryInfC{$\Gamma_3, C, \Gamma'_3 \vdash \Delta_3$}
					
					\RightLabel{Cut}
					\BinaryInfC{$\Gamma_3, \Gamma_1, A \parr B, \Gamma_2, \Gamma'_3 \vdash \Delta_1, \Delta_3, \Delta'_1 \Delta_2$}
					\DisplayProof
				}
				&
				$\overset{(\downarrow \parr_L, -)}{\rightsquigarrow}$
				&
				\scalebox{0.6}{
					\AxiomC{$\pi_1$}
					\UnaryInfC{$\Gamma_1, A \vdash \Delta_1, C, \Delta'_1 $}
					
					\AxiomC{$\tau_2$}
					\noLine
					\UnaryInfC{$\svdots$}
					\UnaryInfC{$\Gamma_3, C, \Gamma'_3 \vdash \Delta_3$}
					
					\RightLabel{Cut}
					\BinaryInfC{$\Gamma_3,\Gamma_1, A , \Gamma'_3 \vdash \Delta_1, \Delta_3, \Delta'_1 $}
					\doubleLine
					\RightLabel{ex.}
					\UnaryInfC{$\Gamma_3,\Gamma_1, \Gamma'_3, A \vdash \Delta_1, \Delta_3, \Delta'_1 $}
					
					\AxiomC{$\pi_2$}
					\UnaryInfC{$B, \Gamma_2 \vdash \Delta_2 $}
					
					\BinaryInfC{$\Gamma_3, \Gamma_1, \Gamma'_3, A \parr B, \Gamma_2 \vdash \Delta_1, \Delta_3, \Delta'_1 \Delta_2 $}
					\doubleLine
					\RightLabel{ex.}
					\UnaryInfC{$\Gamma_3, \Gamma_1, A \parr B, \Gamma_2, \Gamma'_3 \vdash \Delta_1, \Delta_3, \Delta'_1 \Delta_2 $}
					\DisplayProof
				}
				
				\\
				%...................................
				
				\scalebox{0.6}{
					\AxiomC{$\pi_1$}
					\UnaryInfC{$\Gamma_1, A \vdash \Delta_1 $}
					
					\AxiomC{$\pi_2$}
					\UnaryInfC{$B, \Gamma_2 \vdash \Delta_2, C, \Delta'_2 $}
					
					\RightLabel{$\parr_L$}
					\BinaryInfC{$\Gamma_1, A \parr B, \Gamma_2 \vdash \Delta_1, \Delta_2, C, \Delta'_2 $}
					
					\AxiomC{$\tau_2$}
					\noLine
					\UnaryInfC{$\svdots$}
					\UnaryInfC{$\Gamma_3, C, \Gamma'_3 \vdash \Delta_3$}
					
					\RightLabel{Cut}
					\BinaryInfC{$\Gamma_3, \Gamma_1, A \parr B, \Gamma_2, \Gamma'_3 \vdash \Delta_1, \Delta_2, \Delta_3, \Delta'_2 $}
					\DisplayProof
				}
				&
				$\overset{(\downarrow \parr_L, -)}{\rightsquigarrow}$
				&
				\scalebox{0.6}{
					\AxiomC{$\pi_1$}
					\UnaryInfC{$\Gamma_1, A \vdash \Delta_1 $}
					
					\AxiomC{$\pi_2$}
					\UnaryInfC{$B, \Gamma_2 \vdash \Delta_2, C, \Delta'_2 $}
					
					\AxiomC{$\tau_2$}
					\noLine
					\UnaryInfC{$\svdots$}
					\UnaryInfC{$\Gamma_3, C, \Gamma'_3 \vdash \Delta_3$}
					
					\RightLabel{Cut}
					\BinaryInfC{$\Gamma_3, B, \Gamma_2, \Gamma'_3 \vdash \Delta_2, \Delta_3, \Delta'_2 $}
					\doubleLine
					\RightLabel{ex.}
					\UnaryInfC{$B, \Gamma_3, \Gamma_2, \Gamma'_3 \vdash \Delta_2, \Delta_3, \Delta'_2 $}
					
					\BinaryInfC{$\Gamma_1, A \parr B, \Gamma_3, \Gamma_2, \Gamma'_3 \vdash \Delta_1, \Delta_2, \Delta_3, \Delta'_2 $}
					\doubleLine
					\RightLabel{ex.}
					\UnaryInfC{$\Gamma_3, \Gamma_1, A \parr B, \Gamma_2, \Gamma'_3 \vdash \Delta_1, \Delta_2, \Delta_3, \Delta'_2 $}
					\DisplayProof
				}
				
				\\
				%...................................
				
				\AxiomC{$\pi$}
				\UnaryInfC{$\Psi_1 \vdash \Psi_2, A, \Psi'_2 $}
				\RightLabel{$r$}
				\UnaryInfC{$\Phi_1 \vdash \Phi_2, A, \Phi'_2 $}
				
				\AxiomC{$\tau_2$}
				\noLine
				\UnaryInfC{$\svdots$}
				\UnaryInfC{$\Gamma, A, \Gamma' \vdash \Delta $}
				
				\BinaryInfC{$\Gamma, \Phi_1, \Gamma' \vdash \Phi_2, \Delta, \Phi'_2 $}
				\DisplayProof
				&
				$\overset{(\downarrow r, -)}{\rightsquigarrow}$
				&
				\AxiomC{$\pi$}
				\UnaryInfC{$\Psi_1 \vdash \Psi_2, A, \Psi'_2 $}
				
				\AxiomC{$\tau_2$}
				\noLine
				\UnaryInfC{$\svdots$}
				\UnaryInfC{$\Gamma, A, \Gamma' \vdash \Delta $}
				
				\BinaryInfC{$\Gamma, \Psi_1, \Gamma' \vdash \Psi_2, \Delta, \Psi'_2 $}
				
				\doubleLine
				\RightLabel{$r$}
				\UnaryInfC{$\Gamma, \Phi_1, \Gamma' \vdash \Phi_2, \Delta, \Phi'_2$}
				\DisplayProof
			\end{tabularx}
			
			Here, $r$ is one of the following rules: $\otimes_L, \parr_R, 1_L, \bot_R$, ex.R or ex.L. Note that the exchange rule might be performed on $A$ and a neighboring formula $B$, thus possibly requiring multiple applications of the exchange rule in the transformed proof. This is denoted by the double line.
			
			\item $\tau_2$ does not introduce  the principal formula in its last rule. Dually to the case above, we perform one of the following transformations, pushing $\tau_1$ upwards, and use the induction hypothesis on the resulting subproof.
			
			\begin{tabularx}{\textwidth}{x c x}
				\scalebox{0.6}{
					\AxiomC{$\tau_1$}
					\noLine
					\UnaryInfC{$\svdots$}
					\UnaryInfC{$\Gamma_1, \vdash \Delta_1, C, \Delta'_1 $}
					
					\AxiomC{$\pi_1$}
					\UnaryInfC{$\Gamma_2, C, \Gamma_2', A \vdash \Delta_2$}
					
					\AxiomC{$\pi_2$}
					\UnaryInfC{$B, \Gamma_3\vdash \Delta_3 $}
					
					\RightLabel{$\parr_L$}
					\BinaryInfC{$\Gamma_2, C, \Gamma_2', A \parr B, \Gamma_3 \vdash \Delta_2, \Delta_3 $}
					
					\RightLabel{Cut}
					\BinaryInfC{$\Gamma_2, \Gamma_1, \Gamma_2',  A \parr B, \Gamma_3  \vdash \Delta_1, \Delta_2, \Delta_3, \Delta_1' $}
					\DisplayProof
				}
				&
				$\overset{(-, \downarrow \parr_L)}{\rightsquigarrow}$
				&
				\scalebox{0.6}{
					\AxiomC{$\tau_1$}
					\noLine
					\UnaryInfC{$\svdots$}
					\UnaryInfC{$\Gamma_1, \vdash \Delta_1, C, \Delta'_1 $}
					
					\AxiomC{$\pi_1$}
					\UnaryInfC{$\Gamma_2, C, \Gamma_2', A \vdash \Delta_2$}
					
					\RightLabel{Cut}
					\BinaryInfC{$\Gamma_2, \Gamma_1, \Gamma_2', A \vdash \Delta_1, \Delta_2, \Delta_1' $}
					
					\AxiomC{$\pi_2$}
					\UnaryInfC{$B, \Gamma_3 \vdash \Delta_3 $}
					
					\RightLabel{$\parr_L$}
					\BinaryInfC{$\Gamma_2, \Gamma_1, \Gamma_2', A \parr B, \Gamma_3 \vdash \Delta_1, \Delta_2, \Delta_1', \Delta_3 $}
					
					\doubleLine
					\RightLabel{ex.}
					\UnaryInfC{$\Gamma_2, \Gamma_1, \Gamma_2', A\parr B, \Gamma_3  \vdash \Delta_1, \Delta_2, \Delta_3, \Delta_1' $}
					\DisplayProof
				}
				
				\\
				%...................................
				
				\scalebox{0.6}{
					\AxiomC{$\tau_1$}
					\noLine
					\UnaryInfC{$\svdots$}
					\UnaryInfC{$\Gamma_1, \vdash \Delta_1, C, \Delta'_1 $}
					
					\AxiomC{$\pi_1$}
					\UnaryInfC{$\Gamma_2, A \vdash \Delta_2$}
					
					\AxiomC{$\pi_2$}
					\UnaryInfC{$B, \Gamma_3, C, \Gamma_3' \vdash \Delta_3 $}
					
					\RightLabel{$\parr_L$}
					\BinaryInfC{$\Gamma_2, A \parr B, \Gamma_3, C, \Gamma_3' \vdash \Delta_2, \Delta_3 $}
					
					\RightLabel{Cut}
					\BinaryInfC{$\Gamma_2, A \parr B, \Gamma_3, \Gamma_1, \Gamma_3'  \vdash \Delta_1, \Delta_2, \Delta_3, \Delta_1' $}
					\DisplayProof
				}
				&
				$\overset{(-, \downarrow \parr_L)}{\rightsquigarrow}$
				&
				\scalebox{0.6}{
					\AxiomC{$\pi_1$}
					\UnaryInfC{$\Gamma_2, A \vdash \Delta_2$}
					
					\AxiomC{$\tau_1$}
					\noLine
					\UnaryInfC{$\svdots$}
					\UnaryInfC{$\Gamma_1, \vdash \Delta_1, C, \Delta'_1 $}
					
					\AxiomC{$\pi_2$}
					\UnaryInfC{$B, \Gamma_3, C, \Gamma_3' \vdash \Delta_3 $}
					
					\RightLabel{Cut}
					\BinaryInfC{$B, \Gamma_3, \Gamma_1, \Gamma_3' \vdash \Delta_1, \Delta_3, \Delta_1' $}
					
					\RightLabel{$\parr_L$}
					\BinaryInfC{$\Gamma_2, A \parr B, \Gamma_3, \Gamma_1, \Gamma_3' \vdash \Delta_1, \Delta_2, \Delta_1', \Delta_3 $}
					
					\doubleLine
					\RightLabel{ex.}
					\UnaryInfC{$\Gamma_2, A \parr B, \Gamma_3, \Gamma_1, \Gamma_3'  \vdash \Delta_1, \Delta_2, \Delta_3, \Delta_1' $}
					\DisplayProof
				}
				
				\\
				%...................................
				
				\scalebox{0.6}{
					\AxiomC{$\tau_1$}
					\noLine
					\UnaryInfC{$\svdots$}
					\UnaryInfC{$\Gamma_1, \vdash \Delta_1, C, \Delta'_1 $}
					
					\AxiomC{$\pi_1$}
					\UnaryInfC{$\Gamma_2, C, \Gamma_2' \vdash \Delta_2, A$}
					
					\AxiomC{$\pi_2$}
					\UnaryInfC{$\Gamma_3\vdash B, \Delta_3 $}
					
					\RightLabel{$\otimes_R$}
					\BinaryInfC{$\Gamma_2, C, \Gamma_2', \Gamma_3 \vdash \Delta_2, A \otimes B, \Delta_3 $}
					
					\RightLabel{Cut}
					\BinaryInfC{$\Gamma_2, \Gamma_1, \Gamma_2', \Gamma_3  \vdash \Delta_1, \Delta_2, A \otimes B, \Delta_3, \Delta_1' $}
					\DisplayProof
				}
				&
				$\overset{(-, \downarrow \otimes_R)}{\rightsquigarrow}$
				&
				\scalebox{0.6}{
					\AxiomC{$\tau_1$}
					\noLine
					\UnaryInfC{$\svdots$}
					\UnaryInfC{$\Gamma_1, \vdash \Delta_1, C, \Delta'_1 $}
					
					\AxiomC{$\pi_1$}
					\UnaryInfC{$\Gamma_2, C, \Gamma_2' \vdash \Delta_2, A$}
					
					\RightLabel{Cut}
					\BinaryInfC{$\Gamma_2, \Gamma_1, \Gamma_2' \vdash \Delta_1, \Delta_2, A, \Delta_1' $}
					\doubleLine
					\RightLabel{ex.}
					\UnaryInfC{$\Gamma_2, \Gamma_1, \Gamma_2' \vdash \Delta_1, \Delta_2, \Delta_1', A $}
					
					\AxiomC{$\pi_2$}
					\UnaryInfC{$\Gamma_3\vdash B, \Delta_3 $}
					
					\RightLabel{$\otimes_R$}
					\BinaryInfC{$\Gamma_2, \Gamma_1, \Gamma_2', \Gamma_3 \vdash \Delta_1, \Delta_2, \Delta_1',A \otimes B, \Delta_3 $}
					
					\doubleLine
					\RightLabel{ex.}
					\UnaryInfC{$\Gamma_2, \Gamma_1, \Gamma_2', \Gamma_3  \vdash \Delta_1, \Delta_2, A \otimes B, \Delta_3, \Delta_1' $}
					\DisplayProof
				}
				
				\\
				%...................................
				
				\scalebox{0.6}{
					\AxiomC{$\tau_1$}
					\noLine
					\UnaryInfC{$\svdots$}
					\UnaryInfC{$\Gamma_1, \vdash \Delta_1, C, \Delta'_1 $}
					
					\AxiomC{$\pi_1$}
					\UnaryInfC{$\Gamma_2 \vdash \Delta_2, A$}
					
					\AxiomC{$\pi_2$}
					\UnaryInfC{$\Gamma_3 , C, \Gamma_3' \vdash B, \Delta_3 $}
					
					\RightLabel{$\otimes_R$}
					\BinaryInfC{$\Gamma_2, \Gamma_3, C, \Gamma_3' \vdash \Delta_2, A \otimes B, \Delta_3 $}
					
					\RightLabel{Cut}
					\BinaryInfC{$\Gamma_2, \Gamma_3, \Gamma_1, \Gamma_3' \vdash \Delta_1, \Delta_2, A \otimes B, \Delta_3, \Delta_1' $}
					\DisplayProof
				}
				&
				$\overset{(-, \downarrow \otimes_R)}{\rightsquigarrow}$
				&
				\scalebox{0.6}{
					\AxiomC{$\pi_1$}
					\UnaryInfC{$\Gamma_2, \vdash \Delta_2, A$}
					
					\AxiomC{$\tau_1$}
					\noLine
					\UnaryInfC{$\svdots$}
					\UnaryInfC{$\Gamma_1, \vdash \Delta_1, C, \Delta'_1 $}
					
					\AxiomC{$\pi_2$}
					\UnaryInfC{$\Gamma_3, C, \Gamma_3'  \vdash B, \Delta_3 $}
					
					\RightLabel{Cut}
					\BinaryInfC{$\Gamma_3, \Gamma_1, \Gamma_3' \vdash \Delta_1, B, \Delta_3, \Delta_1' $}
					\doubleLine
					\RightLabel{ex.}
					\UnaryInfC{$\Gamma_3, \Gamma_1, \Gamma_3' \vdash B, \Delta_1, \Delta_3, \Delta_1' $}
					
					\RightLabel{$\otimes_R$}
					\BinaryInfC{$\Gamma_2, \Gamma_3, \Gamma_1, \Gamma_3' \vdash \Delta_2, A \otimes B, \Delta_1, \Delta_3, \Delta_1' $}
					
					\doubleLine
					\RightLabel{ex.}
					\UnaryInfC{$\Gamma_2, \Gamma_3, \Gamma_1, \Gamma_3' \vdash \Delta_1, \Delta_2, A \otimes B, \Delta_3, \Delta_1' $}
					\DisplayProof
				}
				
				\\
				%...................................
				
				
				\AxiomC{$\tau_1$}
				\noLine
				\UnaryInfC{$\svdots$}
				\UnaryInfC{$\Gamma \vdash \Delta, A, \Delta' $}
				
				\AxiomC{$\pi$}
				\UnaryInfC{$\Psi_1, A, \Psi'_1 \vdash \Psi_2 $}
				\RightLabel{$r$}
				\UnaryInfC{$\Phi_1, A, \Phi'_1 \vdash \Phi_2 $}
				
				\BinaryInfC{$\Phi_1, \Gamma, \Phi'_1 \vdash \Delta, \Phi_2, \Delta' $}
				\DisplayProof
				&
				$\overset{(-, \downarrow r)}{\rightsquigarrow}$
				&
				\AxiomC{$\tau_1$}
				\noLine
				\UnaryInfC{$\svdots$}
				\UnaryInfC{$\Gamma \vdash \Delta, A, \Delta' $}
				
				\AxiomC{$\pi$}
				\UnaryInfC{$\Psi_1, A, \Psi'_1 \vdash \Psi_2 $}
				
				\BinaryInfC{$\Gamma, \Psi_1, \Gamma' \vdash \Psi_2, \Delta, \Psi'_2 $}
				
				\doubleLine
				\RightLabel{$r$}
				\UnaryInfC{$\Phi_1, \Gamma, \Phi'_1 \vdash \Delta, \Phi_2, \Delta' $}
				\DisplayProof
			\end{tabularx}
			
			Again, $r$ is one of the following rules: $\otimes_L, \parr_R, 1_L, \bot_R$, ex.R or ex.L.
			
			\item Finally, if both $\tau_1$ and $\tau_2$ introduce the principle formula with their last rule, we perform one the following transformations resulting in a proof of lower degree:
			
			\begin{tabularx}{\textwidth}{x c x}
				%cut otimes intro vs otimes intro
				\scalebox{0.6}{
					\AxiomC{$\pi_1$}
					\noLine
					\UnaryInfC{$\svdots$}
					\UnaryInfC{$\Gamma_1 \vdash \Delta_1, A$}
					
					\AxiomC{$\pi_2$}
					\noLine
					\UnaryInfC{$\svdots$}
					\UnaryInfC{$\Gamma_2 \vdash B, \Delta_2$}
					
					\RightLabel{$\otimes_R$}
					\BinaryInfC{$\Gamma_1, \Gamma_2 \vdash \Delta_1, A\otimes B, \Delta_2 $}
					
					\AxiomC{$\pi_3$}
					\noLine
					\UnaryInfC{$\svdots$}
					\UnaryInfC{$\Gamma_3, A, B, \Gamma'_3 \vdash \Delta_3$}
					\RightLabel{$\otimes_L$}
					\UnaryInfC{$\Gamma_3, A \otimes B, \Gamma'_3 \vdash \Delta_3 $}
					
					\RightLabel{Cut}
					\BinaryInfC{$\Gamma_3, \Gamma_1, \Gamma_2, \Gamma'_3 \vdash \Delta_1, \Delta_3, \Delta_2 $}
					\DisplayProof
				}
				&
				$\overset{(\otimes_L, \otimes_R)}{\rightsquigarrow}$
				&
				\scalebox{0.6}{
					\AxiomC{$\pi_1$}
					\noLine
					\UnaryInfC{$\svdots$}
					\UnaryInfC{$\Gamma_1 \vdash \Delta_1, A$}
					
					\AxiomC{$\pi_2$}
					\noLine
					\UnaryInfC{$\svdots$}
					\UnaryInfC{$\Gamma_2 \vdash B, \Delta_2$}
					
					\AxiomC{$\pi_3$}
					\noLine
					\UnaryInfC{$\svdots$}
					\UnaryInfC{$\Gamma_3, A, B, \Gamma'_3 \vdash \Delta_3$}
					
					\RightLabel{Cut}
					\BinaryInfC{$\Gamma_3, A, \Gamma_2, \Gamma'_3 \vdash \Delta_2, \Delta_3 $}
					
					\RightLabel{Cut}
					\BinaryInfC{$\Gamma_3, \Gamma_1, \Gamma_2, \Gamma'_3 \vdash \Delta_1, \Delta_3, \Delta_2 $}
					\DisplayProof
				}
				
				\\
				%.............................................
				
				%cut par intro vs par intro
				\scalebox{0.6}{			
					\AxiomC{$\pi_1$}
					\noLine
					\UnaryInfC{$\svdots$}
					\UnaryInfC{$\Gamma_1 \vdash \Delta_1, A, B, \Delta'_1 $}
					\RightLabel{$\parr_R$}
					\UnaryInfC{$\Gamma_1 \vdash \Delta_1, A \parr B, \Delta'_1 $}
					
					\AxiomC{$\pi_2$}
					\noLine
					\UnaryInfC{$\svdots$}
					\UnaryInfC{$\Gamma_2, A \vdash \Delta_2$}
					
					\AxiomC{$\pi_3$}
					\noLine
					\UnaryInfC{$\svdots$}
					\UnaryInfC{$B, \Gamma_3 \vdash \Delta_3 $}
					
					\RightLabel{$\parr_L$}
					\BinaryInfC{$\Gamma_2, A \parr B, \Gamma_3 \vdash \Delta_2, \Delta_3 $}
					
					\RightLabel{Cut}
					\BinaryInfC{$\Gamma_2, \Gamma_1, \Gamma_3 \vdash \Delta_1, \Delta_2, \Delta_3, \Delta'_1 $}
					\DisplayProof
				}
				&
				$\overset{(\parr_R, \parr_L)}{\rightsquigarrow} $
				&
				\scalebox{0.6}{
					\AxiomC{$\pi_1$}
					\noLine
					\UnaryInfC{$\svdots$}
					\UnaryInfC{$\Gamma_1 \vdash \Delta_1, A, B, \Delta'_1 $}
					
					\AxiomC{$\pi_2$}
					\noLine
					\UnaryInfC{$\svdots$}
					\UnaryInfC{$\Gamma_2, A \vdash \Delta_2$}
					
					\RightLabel{Cut}
					\BinaryInfC{$\Gamma_1, \Gamma_2 \vdash \Delta_1, \Delta_2, B, \Delta'_1 $}
					
					\AxiomC{$\pi_3$}
					\noLine
					\UnaryInfC{$\svdots$}
					\UnaryInfC{$B, \Gamma_3 \vdash \Delta_3 $}
					
					\RightLabel{Cut}
					\BinaryInfC{$\Gamma_1, \Gamma_2, \Gamma_3 \vdash \Delta_1, \Delta_2, \Delta_3, \Delta'_1 $}
					\DisplayProof
				}
				
				\\
				%................................
				
				\AxiomC{\strut}
				\UnaryInfC{$\vdash 1$}
				
				\AxiomC{$\pi$}
				\noLine
				\UnaryInfC{$\svdots$}
				\UnaryInfC{$\Gamma \vdash \Delta $}
				\UnaryInfC{$\Gamma, 1 \vdash \Delta$}
				
				\BinaryInfC{$\Gamma \vdash \Delta$}
				\DisplayProof
				&
				$\overset{(1_R, 1_L)}{\rightsquigarrow}$
				&
				\AxiomC{$\pi$}
				\noLine
				\UnaryInfC{$\svdots$}
				\UnaryInfC{$\Gamma \vdash \Delta $}
				\DisplayProof
				
				\\
				%..............................
				
				\AxiomC{\strut}
				\UnaryInfC{$\bot \vdash $}
				
				\AxiomC{$\pi$}
				\noLine
				\UnaryInfC{$\svdots$}
				\UnaryInfC{$\Gamma \vdash \Delta $}
				\UnaryInfC{$\Gamma \vdash \bot, \Delta$}
				
				\BinaryInfC{$\Gamma \vdash \Delta$}
				\DisplayProof
				&
				$\overset{(\bot_L, \bot_R)}{\rightsquigarrow}$
				&
				
				\AxiomC{$\pi$}
				\noLine
				\UnaryInfC{$\svdots$}
				\UnaryInfC{$\Gamma \vdash \Delta $}
				\DisplayProof
			\end{tabularx}
			
		\end{itemize}
		
		\phantom{bla}
%		\formatnote[]{proofsymbol} testing for proofsymbol
	\end{proof}	
	
	
	\begin{lemma}
		\label{cut-reduction}
		Let $\tau$ be a proof of the sequent $\Gamma \vdash \Delta $ of degree $\partial(\pi)>0$. Then we can construct a proof $\tau'$ with $\partial(\tau')<\partial(\tau)$.
	\end{lemma}
	
	\begin{proof}
		Induction on $h(\tau)$.\\ 
		If the last rule of $\tau$ is not a cut of degree equal $\partial(\tau)$, we gain our proof by applying the induction hypothesis on the immediate subproofs of $\tau$.\\
		If the last rule of $\tau$ is a cut of degree equal to $\partial(\tau)$, we apply the induction hypothesis on the immediate subproofs of $\tau$ and obtain a proof of the following form
		\begin{center}
			\begin{center}
				$\tau = $ 
				\AxiomC{$\tau_1$} 
				\noLine
				\UnaryInfC{$\svdots$}
				\UnaryInfC{$\Gamma_1 \vdash \Delta_1, A, \Delta'_1 $} 
				\AxiomC{$\tau_2$} 
				\noLine
				\UnaryInfC{$\svdots$}
				\UnaryInfC{$\Gamma_2, A, \Gamma'_2 \vdash \Delta_2 $} 
				\BinaryInfC{$\Gamma_2, \Gamma_1, \Gamma'_2 \vdash \Delta_1, \Delta_2, \Delta'_1 $}
				\DisplayProof
			\end{center}
		\end{center}
		with $\partial(\tau_1), \partial(\tau_2) < \partial(A)$. Note that $\Gamma = \Gamma_2\Vert \Gamma_1\Vert \Gamma'_2$ and $\Delta = \Gamma_2\Vert \Gamma_1\Vert \Gamma'_2$,
%		\notimportantnote{just use commata instead of conc.?}
		with $\Vert$ being the concatenation operator.
		We then apply Lemma \ref{key-cases} to the proof tree.
	\end{proof}
	
	With this we now obtain Gantzen's Hauptsatz:
	
	\begin{theorem}[Cut elimination]
		\label{cut-elimination-theorem}
		Given any proof of a sequent we can construct a cut-free proof of that same sequent.
	\end{theorem}
	
	\begin{proof}
		Iteration of Lemma \ref{cut-reduction} on a non-cut-free proof.
	\end{proof}
	
	
	
	%___________________________________________________________________________________________________________________________________________________________________________________________________________________________________________________________________________________________________________________________________________________
	
	\section{Categorical Semantics of Linear Logic}
	%Cats as LinLog semantics
	
	
	
%	\subsection{Proof invariants}
		
	\begin{definition}[$\eta$-expansion]
		We call the following proof transformations \emph{$\eta$-expansions}: 
		\begin{tabularx}{\textwidth}{x c x}
			\AxiomC{\strut}
			\RightLabel{$\id$}
			\UnaryInfC{$A \otimes B \vdash A \otimes B $}
			\DisplayProof
			&
			$\rightsquigarrow$
			&
			\AxiomC{\strut}
			\RightLabel{$\id$}
			\UnaryInfC{$A \vdash A $}
			
			\AxiomC{\strut}
			\RightLabel{$\id$}
			\UnaryInfC{$B \vdash B $}
			
			\RightLabel{$\otimes_R $}
			\BinaryInfC{$A, B \vdash A \otimes B$}
			
			\RightLabel{$\otimes_L $}
			\UnaryInfC{$A \otimes B \vdash A \otimes B $}
			\DisplayProof
			\\
			%..........................................
			\AxiomC{\strut}
			\RightLabel{$\id$}
			\UnaryInfC{$A \parr B \vdash A \parr B $}
			\DisplayProof
			&
			$\rightsquigarrow$
			&
			\AxiomC{\strut}
			\RightLabel{$\id$}
			\UnaryInfC{$A \vdash A $}
			
			\AxiomC{\strut}
			\RightLabel{$\id$}
			\UnaryInfC{$B \vdash B $}
			
			\RightLabel{$\parr_L $}
			\BinaryInfC{$A \parr B \vdash A, B$}
			
			\RightLabel{$\parr_R $}
			\UnaryInfC{$A \parr B \vdash A \parr B $}
			\DisplayProof
			\\
			%.....................................
			\AxiomC{\strut}
			\RightLabel{$\id$}
			\UnaryInfC{$1 \vdash 1 $}
			\DisplayProof
			
			&
			$\rightsquigarrow$
			&
			\AxiomC{\strut}
			\RightLabel{$1_R $}
			\UnaryInfC{$\vdash 1 $}
			\RightLabel{$1_L $}
			\UnaryInfC{$1 \vdash 1 $}
			\DisplayProof
			\\
			\AxiomC{\strut}
			\RightLabel{$\id$}
			\UnaryInfC{$\bot \vdash \bot $}
			\DisplayProof
			&
			$\rightsquigarrow$
			&
			\AxiomC{\strut}
			\RightLabel{$\bot_L $}
			\UnaryInfC{$\bot \vdash $}
			\RightLabel{$\bot_R $}
			\UnaryInfC{$\bot \vdash \bot $}
			\DisplayProof
		\end{tabularx}
	\end{definition}
	
	\begin{definition}
		From now on, when talking about "cut elimination procedure" or "modulo cut elimination", we mean the above mentioned transformations, the proof transformations from Lemma \ref{key-cases} as well as the following commuting cuts:
			
		\begin{tabularx}{\textwidth}{X c X}
			\scalebox{.6}{
				\AxiomC{$\pi_1 $}
				\noLine
				\UnaryInfC{$\svdots$}
				\UnaryInfC{$\Gamma_1 \vdash \Delta_1, A, \Delta'_1 $}
				
				\AxiomC{$\pi_2 $}
				\noLine
				\UnaryInfC{$\svdots$}
				\UnaryInfC{$\Gamma_2, A, \Gamma'_2 \vdash \Delta_2, B, \Delta'_2 $}
				
				\AxiomC{$\pi_3 $}
				\noLine
				\UnaryInfC{$\svdots$}
				\UnaryInfC{$\Gamma_3, B, \Gamma'_3 \vdash \Delta_3 $}
				
				\RightLabel{Cut}
				\BinaryInfC{$\Gamma_3, \Gamma_2, A, \Gamma'_2, \Gamma'_3 \vdash \Delta_2, \Delta_3, \Delta'_2 $}
				
				\RightLabel{Cut}
				\BinaryInfC{$\Gamma_3, \Gamma_2, \Gamma_1, \Gamma'_2, \Gamma'_3 \vdash \Delta_1, \Delta_2, \Delta_3, \Delta'_2, \Delta'_1 $}
				\DisplayProof
			}
			&
			$\leftrightsquigarrow$
			&
			\scalebox{.6}{
				\AxiomC{$\pi_1 $}
				\noLine
				\UnaryInfC{$\svdots$}
				\UnaryInfC{$\Gamma_1 \vdash \Delta_1, A, \Delta'_1 $}
				
				\AxiomC{$\pi_2 $}
				\noLine
				\UnaryInfC{$\svdots$}
				\UnaryInfC{$\Gamma_2, A, \Gamma'_2 \vdash \Delta_2, B, \Delta'_2 $}
				
				\RightLabel{Cut}
				\BinaryInfC{$\Gamma_2, \Gamma_1, \Gamma'_2 \vdash \Delta_2, B, \Delta'_2 $}
				
				\AxiomC{$\pi_3 $}
				\noLine
				\UnaryInfC{$\svdots$}
				\UnaryInfC{$\Gamma_3, B, \Gamma'_3 \vdash \Delta_3 $}
				
				\RightLabel{Cut}
				\BinaryInfC{$\Gamma_3, \Gamma_2, \Gamma_1, \Gamma'_2, \Gamma'_3 \vdash \Delta_1, \Delta_2, \Delta_3, \Delta'_2, \Delta'_1 $}
				\DisplayProof
			}
			
			\\
			%..............................................................
			
			\scalebox{.6}{
				\AxiomC{$\pi_1 $}
				\noLine
				\UnaryInfC{$\svdots$}
				\UnaryInfC{$\Gamma_1 \vdash \Delta_1, A, \Delta'_1 $}
				
				\AxiomC{$\pi_2 $}
				\noLine
				\UnaryInfC{$\svdots$}
				\UnaryInfC{$\Gamma_2 \vdash \Delta_2, B, \Delta'_2 $}
				
				\AxiomC{$\pi_3 $}
				\noLine
				\UnaryInfC{$\svdots$}
				\UnaryInfC{$\Gamma_3, A, B, \Gamma'_3 \vdash \Delta_3 $}
				
				\RightLabel{Cut}
				\BinaryInfC{$\Gamma_3, A, \Gamma_2, \Gamma'_3 \vdash \Delta_2, \Delta_3, \Delta'_2 $}
				
				\RightLabel{Cut}
				\BinaryInfC{$\Gamma_3, \Gamma_1, \Gamma_2, \Gamma'_3 \vdash \Delta_1, \Delta_2, \Delta_3, \Delta'_2, \Delta'_1 $}
				\DisplayProof
			}
			&
			$\leftrightsquigarrow$
			&
			\scalebox{.6}{		
				\AxiomC{$\pi_2 $}
				\noLine
				\UnaryInfC{$\svdots$}
				\UnaryInfC{$\Gamma_2 \vdash \Delta_2, B, \Delta'_2 $}
				
				\AxiomC{$\pi_1 $}
				\noLine
				\UnaryInfC{$\svdots$}
				\UnaryInfC{$\Gamma_1 \vdash \Delta_1, A, \Delta'_1 $}
				
				\AxiomC{$\pi_3 $}
				\noLine
				\UnaryInfC{$\svdots$}
				\UnaryInfC{$\Gamma_3, A, B, \Gamma'_3 \vdash \Delta_3 $}
				
				\RightLabel{Cut}
				\BinaryInfC{$\Gamma_3, \Gamma_1, B, \Gamma'_3 \vdash \Delta_1, \Delta_3, \Delta'_1 $}
				
				\RightLabel{Cut}
				\BinaryInfC{$\Gamma_3, \Gamma_1, \Gamma_2, \Gamma'_3 \vdash \Delta_2, \Delta_1, \Delta_3, \Delta'_1, \Delta'_2 $}
				\DisplayProof
			}
			
			\\
			%..............................................................
			
		\end{tabularx}
		
		
%		\urgentnote[inline]{think about the deltas...: aufteilen in zwei Regeln mit einseitigen Deltas für A und B oder allgemein halten und viele exchanges drunter klatschen (was es asymmetrisch macht und damit)?}
		
	\end{definition}
	
	We will now define proof invariants as outlined by Melliès \cite{mellies}.
	
	\begin{definition}[Proof invariant]
		\label{def: invariant}
		Let $\pi$ be a proof tree ending in a sequent of the form $A \vdash B$, i.e. with a single formula on either side of the turnstile. We call any function
		\begin{align*}
			\pi \mapsto [\pi]
		\end{align*}
		that remains constant under cut elimination \emph{proof invariant}. The entity $[\pi]$ is called \emph{denotation} of the proof $\pi$.
	\end{definition}
	
	
	\begin{definition}[Modularity]
		We call a proof invariant \emph{modular} iff there exists a binary operation $\circ$ such that for any two proofs
		\begin{center}
			\AxiomC{$\pi_1 $}
			\noLine
			\UnaryInfC{$\svdots$}
			\UnaryInfC{$A \vdash B$} 
			\DisplayProof
			\quad
			and
			\quad
			\AxiomC{$\pi_2 $}
			\noLine
			\UnaryInfC{$\svdots$}
			\UnaryInfC{$B \vdash C$} 
			\DisplayProof
		\end{center}
		the denotation of the proof
		\begin{center}
			$\pi$ = 
			\AxiomC{$\pi_1 $}
			\noLine
			\UnaryInfC{$\svdots$}
			\UnaryInfC{$A \vdash B$} 
			\AxiomC{$\pi_2 $}
			\noLine
			\UnaryInfC{$\svdots$}
			\UnaryInfC{$B \vdash C$} 
			\RightLabel{Cut}
			\BinaryInfC{$A \vdash C$}
			\DisplayProof
		\end{center}
		is given by $[\pi] = [\pi_2] \circ [\pi_1]$. As the symbol suggests, we call this operation \emph{composition}.
	\end{definition}
	
	
	\begin{proposition}
		A modular invariant of proofs forms a category with the formulae as objects and the proof denotations as morphisms.
	\end{proposition}
	
	\begin{proof}
		Associativity is given by commuting cuts. The identity morphisms are given by the axiom rule. 
	\end{proof}
	
	
	\begin{definition}[tensoriality]
		A proof invariant is called \emph{tensorial} iff there is a binary operation $\otimes$ such that for any two proofs
		\AxiomC{$\pi_1 $}
		\noLine
		\UnaryInfC{$\svdots$}
		\UnaryInfC{$A \vdash C$} 
		\DisplayProof
		and
		\AxiomC{$\pi_2 $}
		\noLine
		\UnaryInfC{$\svdots$}
		\UnaryInfC{$B \vdash D$} 
		\DisplayProof
		the denotation of the proof
%		\urgentnote{what about the proof without $\otimes_L$?}
%comment.................................................................................................................................
		\begin{center}
			$\pi$ = 
			\AxiomC{$\pi_1 $}
			\noLine
			\UnaryInfC{$\svdots$}
			\UnaryInfC{$A \vdash C$} 
			\AxiomC{$\pi_2 $}
			\noLine
			\UnaryInfC{$\svdots$}
			\UnaryInfC{$B \vdash D$} 
			\RightLabel{$\otimes_R$}
			\BinaryInfC{$A, B \vdash C \otimes D$}
			\RightLabel{$\otimes_L$}
			\UnaryInfC{$A \otimes B \vdash C \otimes D$}
			\DisplayProof
		\end{center}
		is given by $[\pi] = [\pi_1] \otimes [\pi_2]$.
	\end{definition}
	
	
	\begin{definition}[cotensorial]
		A proof invariant is called \emph{cotensorial} iff there is a binary operation $\otimes$ such that for any two proofs
		\AxiomC{$\pi_1 $}
		\noLine
		\UnaryInfC{$\svdots$}
		\UnaryInfC{$A \vdash C$} 
		\DisplayProof
		and
		\AxiomC{$\pi_2 $}
		\noLine
		\UnaryInfC{$\svdots$}
		\UnaryInfC{$B \vdash D$} 
		\DisplayProof
		the denotation of the proof
%		\urgentnote{what about the proof without $\parr_R$?}
%comment................................................................................................................................................
		\begin{center}
			$\pi$ = 
			\AxiomC{$\pi_1 $}
			\noLine
			\UnaryInfC{$\svdots$}
			\UnaryInfC{$A \vdash C$} 
			\AxiomC{$\pi_2 $}
			\noLine
			\UnaryInfC{$\svdots$}
			\UnaryInfC{$B \vdash D$} 
			\RightLabel{$\parr_L$}
			\BinaryInfC{$A \parr B \vdash C, D$}
			\RightLabel{$\parr_R$}
			\UnaryInfC{$A \parr B \vdash C \parr D$}
			\DisplayProof
		\end{center}
		is given by $[\pi] = [\pi_1] \parr [\pi_2]$.
	\end{definition}
	
	
	\begin{lemma}
		A (co-)tensorial, modular proof invariant forms a symmetric monoidal category.
	\end{lemma}
	
	\begin{proof}
		We focus on tensorial proof invariants as the proof trees for the cotensorial case are geometrically the same. From now on we abuse our notation and write $\pi$ for the denotation $[\pi]$ of the proof $\pi$.
		
		It is easy to see that $\otimes$ forms a bifunctor:
		By $\eta$-expansion the two proofs 
		\begin{center}
			\AxiomC{\strut}
			\RightLabel{$\id$}
			\UnaryInfC{$A \otimes B \vdash A \otimes B $}
			\DisplayProof
			\quad
			and 
			\quad
			\AxiomC{\strut}
			\RightLabel{$\id$}
			\UnaryInfC{$A \vdash A $}
			\AxiomC{\strut}
			\RightLabel{$\id$}
			\UnaryInfC{$B \vdash B $}
			\RightLabel{$\otimes_R $}
			\BinaryInfC{$A, B \vdash A \otimes B$}
			\RightLabel{$\otimes_L $}
			\UnaryInfC{$A \otimes B \vdash A \otimes B $}
			\DisplayProof
		\end{center}
		have the same denotation $\id_{A \otimes B} = \id_A \otimes \id_B$.
		
		Given four proofs
		\begin{center}
			\AxiomC{$\pi_1 $}
			\noLine
			\UnaryInfC{$\svdots$}
			\UnaryInfC{$A_1 \vdash A_2 $}
			\DisplayProof
			, \quad
			\AxiomC{$\pi_2 $}
			\noLine
			\UnaryInfC{$\svdots$}
			\UnaryInfC{$B_1 \vdash B_2$}
			\DisplayProof
			, \quad
			\AxiomC{$\pi_3 $}
			\noLine
			\UnaryInfC{$\svdots$}
			\UnaryInfC{$A_2 \vdash A_3$}
			\DisplayProof
			\quad
			and 
			\quad
			\AxiomC{$\pi_4 $}
			\noLine
			\UnaryInfC{$\svdots$}
			\UnaryInfC{$B_2 \vdash B_3$}
			\DisplayProof
		\end{center}
		
		the proof tree
		\begin{center}
			\AxiomC{$\pi_1 $}
			\noLine
			\UnaryInfC{$\svdots$}
			\UnaryInfC{$A_1 \vdash A_2 $}
			
			\AxiomC{$\pi_2 $}
			\noLine
			\UnaryInfC{$\svdots$}
			\UnaryInfC{$B_1 \vdash B_2$}
			
			\RightLabel{$\otimes_R$}
			\BinaryInfC{$A_1, B_1 \vdash A_2, B_2 $}
			\RightLabel{$\otimes_L$}
			\UnaryInfC{$A_1 \otimes B_1 \vdash A_2, B_2 $}
			
			\AxiomC{$\pi_3 $}
			\noLine
			\UnaryInfC{$\svdots$}
			\UnaryInfC{$A_2 \vdash A_3$}
			
			\AxiomC{$\pi_4 $}
			\noLine
			\UnaryInfC{$\svdots$}
			\UnaryInfC{$B_2 \vdash B_3$}
			
			\RightLabel{$\otimes_R$}
			\BinaryInfC{$A_2, B_2 \vdash A_3 \otimes B_3 $}
			\RightLabel{$\otimes_L$}
			\UnaryInfC{$A_2 \otimes B_2 \vdash A_3 \otimes B_3 $}
			
			\RightLabel{Cut}
			\BinaryInfC{$A_1 \otimes B_1 \vdash A_3 \otimes B_3 $}
			
			\DisplayProof
		\end{center}
		
		with denotation $(\pi_3 \otimes \pi_4) \circ (\pi_1 \otimes \pi_2)$
		is transformed via cut elimination into the proof
		\begin{center}
			\AxiomC{$\pi_1 $}
			\noLine
			\UnaryInfC{$\svdots$}
			\UnaryInfC{$A_1 \vdash A_2 $}
			
			\AxiomC{$\pi_3 $}
			\noLine
			\UnaryInfC{$\svdots$}
			\UnaryInfC{$A_2 \vdash A_3$}
			
			\RightLabel{Cut}
			\BinaryInfC{$A_1 \vdash A_3 $}
			
			\AxiomC{$\pi_2 $}
			\noLine
			\UnaryInfC{$\svdots$}
			\UnaryInfC{$B_1 \vdash B_2$}
			
			\AxiomC{$\pi_4 $}
			\noLine
			\UnaryInfC{$\svdots$}
			\UnaryInfC{$B_2 \vdash B_3$}
			
			\RightLabel{Cut}
			\BinaryInfC{$B_1 \vdash B_3 $}
			
			\RightLabel{$\otimes_R$}
			\BinaryInfC{$A_1, B_1 \vdash A_3 \otimes B_3 $}
			
			\RightLabel{$\otimes_L$}
			\UnaryInfC{$A_1 \otimes B_1 \vdash A_3 \otimes B_3 $}
			
			\DisplayProof
		\end{center}
		
		with denotation $(\pi_3\circ \pi_1) \otimes (\pi_4 \circ \pi_2)$ which gives us the following equation
		\begin{align*}
			(\pi_3 \otimes \pi_4) \circ (\pi_1 \otimes \pi_2)
			= (\pi_3\circ \pi_1) \otimes (\pi_4 \circ \pi_2)
		\end{align*}
		
		The associator $\alpha$ is given by the denotation of the proof
		\begin{center}
			\AxiomC{\strut}
			\RightLabel{$\id$}
			\UnaryInfC{$A \vdash A $}
			
			\AxiomC{\strut}
			\RightLabel{$\id$}
			\UnaryInfC{$B \vdash B $}
			
			\AxiomC{\strut}
			\RightLabel{$\id$}
			\UnaryInfC{$C \vdash C $}
			
			\RightLabel{$\otimes_R $}
			\BinaryInfC{$B, C \vdash B \otimes C$}
			
			\RightLabel{$\otimes_R$}
			\BinaryInfC{$A, B, C \vdash A \otimes (B \otimes C )$}
			
			\RightLabel{$\otimes_L $}
			\UnaryInfC{$A \otimes B, C \vdash A \otimes (B \otimes C )$}
			\RightLabel{$\otimes_L $}
			\UnaryInfC{$(A \otimes B) \otimes C \vdash A \otimes (B \otimes C) $}
			\DisplayProof
		\end{center}
		
		while the unitors $\lambda$ and $\rho$ are given by
		\begin{center}
			\AxiomC{\strut}
			\RightLabel{$\id$}
			\UnaryInfC{$A \vdash A $}
			\RightLabel{$1_L$}
			\UnaryInfC{$1, A\vdash A$}
			\UnaryInfC{$1 \otimes A \vdash A$}
			\DisplayProof
			\quad
			and
			\quad
			\AxiomC{\strut}
			\RightLabel{$\id$}
			\UnaryInfC{$A \vdash A $}
			\RightLabel{$1_L$}
			\UnaryInfC{$A, 1\vdash A$}
			\UnaryInfC{$A \otimes 1\vdash A$}
			\DisplayProof
		\end{center}
		We have to show isomorphy as well as the naturality and coherence conditions.
		The inverse morphisms are given by 
		
		\begin{center}
			\AxiomC{\strut}
			\RightLabel{$\id$}
			\UnaryInfC{$A \vdash A $}
			
			\AxiomC{\strut}
			\RightLabel{$\id$}
			\UnaryInfC{$B \vdash B $}
			
			\RightLabel{$\otimes_R $}
			\BinaryInfC{$A, B \vdash A \otimes B$}
			
			\AxiomC{\strut}
			\RightLabel{$\id$}
			\UnaryInfC{$C \vdash C $}
			
			\RightLabel{$\otimes_R$}
			\BinaryInfC{$A, B, C \vdash (A \otimes B) \otimes C $}
			
			\RightLabel{$\otimes_L $}
			\UnaryInfC{$A , B \otimes C \vdash  (A \otimes B) \otimes C $}
			\RightLabel{$\otimes_L $}
			\UnaryInfC{$A \otimes (B \otimes C) \vdash  (A \otimes B) \otimes C  $}
			\DisplayProof
		\end{center}
		
		as well as
		
		\begin{center}
			\AxiomC{\strut}
			\RightLabel{$1_R$}
			\UnaryInfC{$\vdash 1$}
%			
			\AxiomC{\strut}
			\RightLabel{$\id$}
			\UnaryInfC{$A \vdash A $}
%
			\RightLabel{$\otimes_R$}
			\BinaryInfC{$A \vdash 1 \otimes A$}
			\DisplayProof
			\quad 
			and 
			\quad
%
			\AxiomC{\strut}
			\RightLabel{$\id$}
			\UnaryInfC{$A \vdash A $}
%			
			\AxiomC{\strut}
			\RightLabel{$1_R$}
			\UnaryInfC{$\vdash 1$}
%
			\RightLabel{$\otimes_R$}
			\BinaryInfC{$A \vdash A \otimes 1$}
			\DisplayProof
		\end{center}
		
		The equation $\lambda\circ\lambda^{-1} = \id_A $ is easily shown:
		
		\begin{tabularx}{\textwidth}{c x c x}
			
			&
			\scalebox{.8}{
				\AxiomC{\strut}
				\RightLabel{$1_R$}
				\UnaryInfC{$\vdash 1$}
				
				\AxiomC{\strut}
				\RightLabel{$\id$}
				\UnaryInfC{$A \vdash A $}
				
				\RightLabel{$\otimes_R$}
				\BinaryInfC{$A \vdash 1 \otimes A$}
				
				\AxiomC{\strut}
				\RightLabel{$\id$}
				\UnaryInfC{$A \vdash A $}
				
				\RightLabel{$1_L$}
				\UnaryInfC{$1, A\vdash A$}
				
				\RightLabel{$\otimes_L$}
				\UnaryInfC{$1 \otimes A \vdash A$}
				
				\RightLabel{Cut}
				\BinaryInfC{$A \vdash A $}
				\DisplayProof
			}
			&
			$\overset{(\otimes_R, \otimes_L)}{\rightsquigarrow}$
			&
			\scalebox{.8}{
				\AxiomC{\strut}
				\RightLabel{$1_R$}
				\UnaryInfC{$\vdash 1$}
				
				\AxiomC{\strut}
				\RightLabel{$\id$}
				\UnaryInfC{$A \vdash A $}
				
				\AxiomC{\strut}
				\RightLabel{$\id$}
				\UnaryInfC{$A \vdash A $}
				
				\RightLabel{$1_L$}
				\UnaryInfC{$1, A\vdash A$}
				
				\RightLabel{Cut}
				\BinaryInfC{$1, A \vdash A $}
				
				\RightLabel{Cut}
				\BinaryInfC{$A \vdash A $}
				\DisplayProof
			}
			%......................................
			\\
			
			
			$\overset{(-, \id)}{\rightsquigarrow}$
			&
			
			\scalebox{.8}{
				\AxiomC{\strut}
				\RightLabel{$1_R$}
				\UnaryInfC{$\vdash 1$}
				
				\AxiomC{\strut}
				\RightLabel{$\id$}
				\UnaryInfC{$A \vdash A $}
				
				\RightLabel{$1_L$}
				\UnaryInfC{$1, A\vdash A$}
				
				\RightLabel{Cut}
				\BinaryInfC{$A \vdash A $}
				\DisplayProof
			}
			
			&
			$\overset{(1_R, 1_L)}{\rightsquigarrow}$
			&
			
			\AxiomC{\strut}
			\RightLabel{$\id$}
			\UnaryInfC{$A \vdash A $}
			\DisplayProof
			
		\end{tabularx}
		
		The equation $\lambda^{-1} \circ \lambda = \id_{1 \otimes A} $ requires additional steps. First, we eliminate the cut in $\lambda^{-1} \circ \lambda$:
		
		\begin{tabularx}{\textwidth}{x c x}
			\scalebox{.8}{
				\AxiomC{\strut}
				\RightLabel{$\id$}
				\UnaryInfC{$A \vdash A $}
				
				\RightLabel{$1_L$}
				\UnaryInfC{$1, A\vdash A$}
				
				\RightLabel{$\otimes_L$}
				\UnaryInfC{$1 \otimes A \vdash A$}
				
				\AxiomC{\strut}
				\RightLabel{$1_R$}
				\UnaryInfC{$\vdash 1$}
				
				\AxiomC{\strut}
				\RightLabel{$\id$}
				\UnaryInfC{$A \vdash A $}
				
				\RightLabel{$\otimes_R$}
				\BinaryInfC{$A \vdash 1 \otimes A$}
				
				\RightLabel{Cut}
				\BinaryInfC{$1 \otimes A \vdash 1 \otimes A $}
				\DisplayProof
			}
			&
			$\overset{\downarrow \otimes_L, \downarrow \otimes_R, (-, \id) }{\rightsquigarrow}$
			&
			\scalebox{.8}{
				\AxiomC{\strut}
				\RightLabel{$1_R$}
				\UnaryInfC{$\vdash 1$}
				
				\AxiomC{\strut}
				\RightLabel{$\id$}
				\UnaryInfC{$A \vdash A $}
				
				\RightLabel{$1_L$}
				\UnaryInfC{$1, A\vdash A$}
				
				\RightLabel{$\otimes_R$}
				\BinaryInfC{$1, A \vdash 1 \otimes A $}
				
				\RightLabel{$\otimes_L$}
				\UnaryInfC{$1 \otimes A \vdash A$}
				
				\DisplayProof
			}
						
		\end{tabularx}
		
		Then, cut $\lambda^{-1} \circ \lambda$ against the $\eta$-expansion of $\id_{1 \otimes A}$:
		
		\begin{tabularx}{\textwidth}{x c x}
			
			\scalebox{.65}{
				\AxiomC{\strut}
				\RightLabel{$1_R$}
				\UnaryInfC{$\vdash 1$}
				
				\AxiomC{\strut}
				\UnaryInfC{$A \vdash A $}
				
				\RightLabel{$1_L$}
				\UnaryInfC{$1, A\vdash A$}
				
				\RightLabel{$\otimes_R$}
				\BinaryInfC{$1, A \vdash 1 \otimes A $}
				
				\RightLabel{$\otimes_L$}
				\UnaryInfC{$1 \otimes A \vdash A$}
				
				\AxiomC{\strut}
				\UnaryInfC{$1 \vdash 1$}
				
				\AxiomC{\strut}
				\UnaryInfC{$A \vdash A $}
				
				\BinaryInfC{$1, A \vdash 1 \otimes A $}
				
				\RightLabel{$\otimes_L$}
				\UnaryInfC{$1 \otimes A \vdash 1 \otimes A $}
				
				\RightLabel{Cut}
				\BinaryInfC{$1 \otimes A \vdash 1 \otimes A $}
				
				\DisplayProof
			}
			&
			$\overset{(\downarrow \otimes_L, -) , (\otimes_R, \otimes_L)}{\rightsquigarrow}$
			&
			\scalebox{.7}{
				\AxiomC{\strut}
				\RightLabel{$1_R$}
				\UnaryInfC{$\vdash 1$}
				
					\AxiomC{\strut}
					\UnaryInfC{$A \vdash A $}
					\RightLabel{$1_L$}
					\UnaryInfC{$1, A\vdash A$}
					
							\AxiomC{\strut}
							\UnaryInfC{$1 \vdash 1$}
							
							\AxiomC{\strut}
							\UnaryInfC{$A \vdash A $}
							
						\RightLabel{$\otimes_R$}
						\BinaryInfC{$1, A \vdash 1 \otimes A $}
						
				
				\RightLabel{Cut}
				\BinaryInfC{$1, 1, A \vdash 1 \otimes A $}
				
				\RightLabel{Cut}
				\BinaryInfC{$1 , A \vdash 1 \otimes A $}
				
				\RightLabel{$\otimes_L$}
				\UnaryInfC{$1 \otimes A \vdash 1 \otimes A $}
				
				\DisplayProof
			}
			
		\end{tabularx}
		
		%tabularx wollte nicht passend formatieren
		
		\begin{tabularx}{\textwidth}{c x c x}
			
			$\overset{(-, \downarrow \otimes_R) , (-, \id)}{\rightsquigarrow}$
			&
			
			\scalebox{.7}{
				\AxiomC{\strut}
				\RightLabel{$1_R$}
				\UnaryInfC{$\vdash 1$}
				
				\AxiomC{\strut}
				\UnaryInfC{$1 \vdash 1$}
				
				\AxiomC{\strut}
				\UnaryInfC{$A \vdash A $}
				
				\RightLabel{$1_L$}
				\UnaryInfC{$1, A\vdash A$}
				
				\RightLabel{$\otimes_R$}
				\BinaryInfC{$1, 1, A \vdash 1 \otimes A $}
				
				\RightLabel{Cut}
				\BinaryInfC{$1, A \vdash A$}
				
				\RightLabel{$\otimes_L$}
				\UnaryInfC{$1 \otimes A \vdash 1 \otimes A $}
				
				\DisplayProof
			}
			
			&
			$\overset{(-, \downarrow \otimes_R) , (1_R, 1_L)}{\rightsquigarrow}$
			&
			\scalebox{.7}{
				\AxiomC{\strut}
				\UnaryInfC{$1 \vdash 1$}
				
				\AxiomC{\strut}
				\UnaryInfC{$A \vdash A $}
				
				\RightLabel{$\otimes_R$}
				\BinaryInfC{$1, 1, A \vdash 1 \otimes A $}
				
				\RightLabel{$\otimes_L$}
				\UnaryInfC{$1 \otimes A \vdash 1 \otimes A $}
				
				\DisplayProof
			}
			
		\end{tabularx}
		
		Thus, we have 
		$\id_{1 \otimes A} \circ (\lambda^{-1} \circ \lambda) = \id_{1 \otimes A}$ 
		and by modularity 
		$\lambda^{-1} \circ \lambda = \id_{1 \otimes A}$.
		
		Naturality for $\lambda$ means that for any $\pi: A \rightarrow B $ the equation $\lambda_B \circ (\id_1 \otimes \pi) = \pi \circ \lambda_A $ holds.
		Cut elimination on $\lambda_B \circ (\id_1 \otimes \pi)$ gives us:
		
		\begin{tabularx}{\textwidth}{x c x}			
			\scalebox{.8}{
				\AxiomC{\strut}
				\UnaryInfC{$1 \vdash 1$}
				\AxiomC{$\pi $}
				\noLine
				\UnaryInfC{$\svdots$}
				\UnaryInfC{$A \vdash B$} 
				\RightLabel{$\otimes_R$}
				\BinaryInfC{$1, A \vdash 1 \otimes B$}
				\RightLabel{$\otimes_L$}
				\UnaryInfC{$1 \otimes A \vdash 1 \otimes B$}
				
				\AxiomC{\strut}
				\RightLabel{$\id$}
				\UnaryInfC{$B \vdash B $}
				\RightLabel{$1_L$}
				\UnaryInfC{$1, B\vdash B$}
				\RightLabel{$\otimes_L$}
				\UnaryInfC{$1 \otimes B \vdash B$}
				
				\RightLabel{Cut}
				\BinaryInfC{$1 \otimes A \vdash B$}
				
				\DisplayProof
			}
			&
			$\overset{(\downarrow \otimes_L, -) , (\otimes_R, \otimes_L)}{\rightsquigarrow}$
			&
			\scalebox{1}{
				\AxiomC{\strut}
				\UnaryInfC{$1 \vdash 1$}
				
				\AxiomC{$\pi $}
				\noLine
				\UnaryInfC{$\svdots$}
				\UnaryInfC{$A \vdash B$}
				
				\AxiomC{\strut}
				\RightLabel{$\id$}
				\UnaryInfC{$B \vdash B $}
				\RightLabel{$1_L$}
				\UnaryInfC{$1, B\vdash B$}
				 
				
				\RightLabel{Cut}
				\BinaryInfC{$1, A\vdash B$}
				
				\RightLabel{Cut}
				\BinaryInfC{$1, A\vdash B$}
				
				\RightLabel{$\otimes_L$}
				\UnaryInfC{$1 \otimes A \vdash B$}
				
				\DisplayProof
			}
		\end{tabularx}
		
		\begin{tabularx}{\textwidth}{x x x}
			
			$\underset{ (-, \id) }{\overset{(-, \downarrow 1_L)(\id, -)}{\rightsquigarrow}} $
			&
			
			\scalebox{}{
				\AxiomC{$\pi $}
				\noLine
				\UnaryInfC{$\svdots$}
				\UnaryInfC{$A \vdash B$} 
				
				\RightLabel{$1_L$}
				\UnaryInfC{$1, A\vdash B$}
				
				\RightLabel{$\otimes_L$}
				\UnaryInfC{$1 \otimes A \vdash B$}
				
				\DisplayProof
			}
			&
		\end{tabularx}
		
		Which is the same result as cut elimination on $\pi \circ \lambda_A$:
		
		\begin{tabularx}{\textwidth}{x c x}	
			\AxiomC{\strut}
			\RightLabel{$\id$}
			\UnaryInfC{$A \vdash A $}
			\RightLabel{$1_L$}
			\UnaryInfC{$1, A\vdash A$}
			\RightLabel{$\otimes_L$}
			\UnaryInfC{$1 \otimes A \vdash A$}
			
			\AxiomC{$\pi $}
			\noLine
			\UnaryInfC{$\svdots$}
			\UnaryInfC{$A \vdash B$}
			
			\BinaryInfC{$1 \otimes A \vdash B $}
			\DisplayProof
			&
			$\rightsquigarrow$
			&
			\AxiomC{$\pi $}
			\noLine
			\UnaryInfC{$\svdots$}
			\UnaryInfC{$A \vdash B$} 
			
			\RightLabel{$1_L$}
			\UnaryInfC{$1, A\vdash B$}
			
			\RightLabel{$\otimes_L$}
			\UnaryInfC{$1 \otimes A \vdash B$}
			
			\DisplayProof
			
		\end{tabularx}
		
		Naturality and isomorphy of $\alpha$ and $\rho$ as well as the coherence conditions are shown the same way.
	\end{proof}
	
	
	\begin{theorem}
		A proof invariant that is modular, tensorial and cotensorial forms a linearly distributive category.
	\end{theorem}
	
	\begin{proof}
		The distributors $\partial_L$ and $\partial_R$ are given by:
		
		\begin{center}
			\AxiomC{\strut}
			\UnaryInfC{$A \vdash A$}
%			
			\AxiomC{\strut}
			\UnaryInfC{$B \vdash B$}
%			
			\AxiomC{\strut}
			\UnaryInfC{$C \vdash C $}
%			
			\RightLabel{$ \parr_L$}
			\BinaryInfC{$B \parr C \vdash B, C $}
%			
			\RightLabel{$\otimes_R$}
			\BinaryInfC{$A , B \parr C \vdash A \otimes B, C $}
%			
			\RightLabel{$\otimes_L$}
			\UnaryInfC{$ A \otimes (B \parr C ) \vdash A \otimes B , C$}
			\RightLabel{$\parr_R $}
			\UnaryInfC{$ A \otimes (B \parr C ) \vdash (A \otimes B ) \parr C $}
			\DisplayProof
%			
			\quad
			and
			\quad
%			
			\AxiomC{\strut}
			\UnaryInfC{$A \vdash A$}
			%			
			\AxiomC{\strut}
			\UnaryInfC{$B \vdash B$}
			%
			\RightLabel{$ \parr_L$}
			\BinaryInfC{$A \parr B \vdash A, B $}
%						
			\AxiomC{\strut}
			\UnaryInfC{$C \vdash C $}
%			
			\RightLabel{$\otimes_R$}
			\BinaryInfC{$A \parr B , C \vdash A, B \otimes C $}
%			
			\RightLabel{$\otimes_L$}
			\UnaryInfC{$ (A \parr B) \parr C \vdash A , B \otimes C$}
			\RightLabel{$\parr_R $}
			\UnaryInfC{$ (A \parr B) \parr C  \vdash A \parr (B \otimes C) $}
			\DisplayProof
		\end{center}
		
		Coherence and naturality are shown the same way as for the associator and unitors above.
	\end{proof}
	
	
	\begin{remark}
		As noted in Def. \ref{def: invariant}, we have been only dealing with sequents of the form $A \vdash B $. For arbitrary sequents of the form $\Gamma \vdash \Delta $ with $\Gamma = A_1, ... , A_n $ and $\Delta = B_1, ..., B_m $, a natural model are two-tensor-poly-categories as defined by Cockett and Seely \cite{cockett&seely97}. These are (2-categorically) equivalent to linearly distributive categories by identifying $\Gamma$ with $\bigoplus\Gamma = A_1 \oplus \cdots \oplus A_n $ and $\Delta$ with $\bigparr \Delta = B_1 \parr \cdots \parr B_m $ \cite[Theorem 2.1]{cockett&seely97}.
	\end{remark}
	
	
	
	%---------------------------------------------------------------------------------------------------------------------
	
	
	
	
	
	
	
	
	
	
	
	
	
	
	%\todo[inline]{add refs}
	%comment..............................................................................
	\printbibliography
	
%	\newpage
%1	\listoftodos
	
\end{document}
